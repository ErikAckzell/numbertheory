\documentclass{report}


\usepackage[utf8]{inputenc}    
\usepackage[english]{babel}
\usepackage[T1]{fontenc}
\usepackage{amsmath}
\usepackage{amsthm}
\usepackage{amssymb}
\usepackage{mathrsfs}     
\usepackage{comment}
\usepackage{setspace}
\usepackage{color}
\usepackage{graphicx}
\usepackage{fancyhdr}
\usepackage{cancel}
\usepackage{mathrsfs}
\usepackage{enumitem}

\newcommand{\Z}{\mathbb{Z}}
\newcommand{\nequiv}{\cancel{\equiv}}
\newcommand{\up}[1]{\textsuperscript{#1}}
\DeclareMathOperator{\ord}{ord}

\title{Summary: Number Theory}
\begin{document}
\newtheorem*{defi}{Definition}
\newtheorem*{thm}{Theorem}
\subsection*{Chinese remainder theorem}
\[ 
\text{Let  }
\left.
\begin{array}{r c l}
n_1 \dots n_r \in \mathbb{N}\\
a_1 \dots a_r \in \mathbb{Z}\\
\end{array}
\right/ \gcd(n_i,n_j)=1 \forall i \neq j
\]
then the  system
$
\left \{
\begin{array}{r c l}
x &\equiv& a_1[n_1]\\
x &\equiv& a_2[n_2]\\
\vdots &\vdots& \vdots\\
x &\equiv& a_r[n_r]\\
\end{array}
\right.
$ \\
has a simultaneous solution which is unique modulo $n_1 \cdot n_2 \dots n_r$
\[
\Rightarrow f(x) \equiv 0[n] \text{ with  } n = p_1^{k_1} \dots p_r^{k_r} \qquad 
\left \{
\begin{array}{r c l}
f(x) &\equiv& 0[p_1^{k_1}]\\
&\vdots&\\
f(x) &\equiv& 0[p_r^{k_r}]\\
\end{array}
\right.
\]
Set 
\[
n=n_1 \dots n_r
\]
And
\[
N_k=\frac{n}{n_k}=n_1 n_2 \dots n_{k-1} n_{k+1} \dots n_r
\]
Then, $N_k x_k \equiv 1 [n_k]$\\
\[
N_k x_k + n_k y_k = 1, \text{exists because } \gcd(N_k, n_k)=1
\]
And if we set $\bar{x} \equiv \sum_{k=1}^{r}a_k N_k x_k \equiv a_k [n_k]$, then $\bar{x}$ is a simultaneous solution.\\

\subsection*{Fermat's theorem}
Let $p$ be a prime, $a \in \mathbb{Z} / \quad p \cancel{\mid} a$. Then,
\[ a^{p-1} \equiv 1 [p] \]

\subsection*{Wilson's theorem}
If $p$ is a prime, then $(p-1)! \equiv -1[p]$

\subsection*{Number and sum of divisors}
\[
\begin{array}{l c l}
&\tau(n)& = \sum_{d \mid n}1 \text{,   the number of positive divisors of n.}\\
&\sigma(n)&= \sum_{d \mid n}d \text{,   the sum of the positive divisors of n.}\\
\end{array}
\]

\begin{thm} Let $n>1$ with $n=p_1^{k_1} \cdots p_r^{k_r}$. Then \\
	\begin{enumerate}
		\item $\tau(n)=(k_1+1) \cdots (k_r+1)$
		\item $\sigma(n)=(\frac{p_1^{k_1+1}-1}{p_1-1})\dots(\frac{p_r^{k_r+1}-1}{p_r-1})$
	\end{enumerate}
\end{thm}

\begin{thm} Let f be a multiplicative number theoretic function and define $F(n)$ by:
	\[F(n)= \sum_{d\mid n} f(d), \qquad n \geq 1 \]
	Then $F$ is also a multiplicative number theoretic function.
\end{thm}

\underline{Corollary :} $\sigma$ and $\tau$ are multiplicative\\

\subsection*{M\"{o}bius function $\mu$}
	\[ \mu(n)= 	\left \{
	\begin{array}{lcl}
	1 \quad \qquad\text{ if } n=1\\
	0 \quad \qquad \text{ if } \exists p \text{ prime such that } p^2 \mid n\\
	(-1)^r \quad \text{ if }n=p_1p_2 \dots p_r \text{,  with distinct primes.}
	\end{array}
	\right.
	\]
\begin{thm}
	$\mu$ is multiplicative.
	\end{thm}

\begin{thm}
	For $n\geq1$,
	\[ \sum_{d \mid n} \mu(d) = \left \{
	\begin{array}{lcl}
	1 \quad \text{if } n=1\\
	0 \quad \text{if }n>1\\
	\end{array}
	\right.
	\]
	\end{thm}
\subsection*{M\"{o}bius inversion formula}
Let $F$ and $f$ be connected by \[F(n)=\sum_{d \mid n} f(d).\]
Then 
\[ f(n)=\sum_{d\mid n} \mu(d) F(\frac{n}{d}) = \sum_{d \mid n} \mu(\frac{n}{d}) F(d)\]
	
\begin{thm}
	Let $f$, $F$ be connected by:
	\[F(n)=\sum_{d \mid n} f(d)\]
	If $F$ is multiplicative, then $f$ is multiplicative.
\end{thm}

\subsection*{Euler's $\varphi$ function} 
\[\varphi(n)=\mid \{ a \in \mathbb{Z} :1 \leq a \leq n,\quad \gcd(a,n)=1 \} \mid \]
\begin{thm}
	For $p$ prime, $k>0$:
	\[\varphi(p^k)=p^k-p^{k-1}=p^k(1-\frac{1}{p})\]
\end{thm}
\begin{thm} $\varphi$ is multiplicative.
\end{thm}
\underline{Lemma :} Let $n>1$, and $\gcd(a, n)=1$. If 
\begin{equation*}
a_1, \dots, a_{\varphi(n)}\in [1, n)
\end{equation*} 
are the positive integers which are relatively prime to $n$, then 
\begin{equation*}
aa_1, \dots, aa_{\varphi(n)}
\end{equation*}
are congruent to $a_1, \dots, a_{\varphi(n)}$ modulo $n$ in some order.
\subsection*{Euler's theorem}
Let $n>1$ and $\gcd(a, n)=1$. Then 
\begin{equation*}
a^{\varphi(n)}\equiv 1 [n]
\end{equation*}
\underline{Lemma :} For $n\geq 1$, let
\begin{equation*}
S_d = \{m: \quad 1\leq m\leq n,\quad \gcd(m, n)=d\}.
\end{equation*}
Then 
\begin{equation*}
\{1, 2, \dots, n\} = \bigcup_{d \mid n}S_d
\end{equation*}
and the union is disjoint.
\subsection*{Gauss' theorem}
For $n\geq 1$, 
\begin{equation*}
\sum_{d\mid n}\varphi(d)=n
\end{equation*}
\begin{thm}
	For $n\geq 1$,
	\begin{equation*}
	\varphi(n) = n\sum_{d\mid n}\frac{\mu(d)}{d}
	\end{equation*}
\end{thm}
\begin{thm} Let $\ord_n(a)=k$, then:
	\[\begin{array}{lcl}
	a^i\equiv a^j[n]\\
	\iff i=j[k]\\
	\end{array}
	\]
\end{thm}
\begin{thm}
	Let $\ord_n(a)=k, h>0$\\
	Then $a^h$ has order $\ord_n(a^h)= \frac{k}{\gcd(h,k)}$\\
\end{thm}

\underline{\textbf{Lagrange's theorem:}} Let p be a prime, and $f$ such that:
\[f(x)=a_nx^n+\dots+a_1x+a_0,\quad a_i\in \mathbb{Z}, \quad a_n \nequiv 0[p]\]
Then  $f(x) \equiv 0[p]$ has at most $n$ incongruent solutions.
\textbf{Corollary:} For $p$ prime, $d\mid p-1$: \[x^d-1\equiv0[p]\]
has exactly $d$ incongruent solutions.
\begin{thm} Let $p$ be a prime, $d\mid p-1$, then there are exactly $\varphi(d)$ incongruent integers of order $d$ modulo p.
\end{thm}
\textbf{Corollary:} A prime $p$ has exactly $\varphi(p-1)$ primitive roots.\\
\underline{Application:} $p\equiv 1[4] \Rightarrow x^2\equiv -1[p]$ has a solution.\\
Take $d=4$ in the theorem: $4 \mid p-1$\\
Then there exists an $a$ of order 4 modulo $p$.\\
\[p \mid (a^4-1)=(a^2-1)(a^2+1)\]
\[p \mid a^2-1 \text{  or  } \mid a^2+1\]
\[\Rightarrow \left. \begin{array}{lcl}
a^2 \equiv 1[p]\\
a^2 \equiv -1[p]
\end{array}
\right \} x=a \text{ is a solution to   } x^2 \equiv -1[p]\]
\underline{Lemma:} $p$ an odd prime, there is a primitive element $r [p]$ such that:\\
\[r^{p-1} \nequiv 1[p^2]\]
\begin{thm} $k\geq 1$, $\forall p$ odd prime, $\exists r[p^k]$ a primitive root.
\end{thm}
\begin{defi} For $\gcd(a,n)=1$, the \textbf{index} of $a$ in the base $r$ is the smallest positive integer $h$ such that:\[a \equiv r^h[n] \]
	\[ind_r(a)=h\]
	If $a\equiv b[n]$ then $ind(a)=ind(b)$
\end{defi}
\begin{thm} $n,r$ as above.
	\begin{itemize}
		\item[a)] $ind_r(a,b)\equiv ind_r(a)+ind_r(b)[\varphi(n)]$
		\item[b)] $ind_r(a^k) \equiv k \cdot ind_r(a)[\varphi(n)]$
		\item[c)] $ind_r(1)\equiv 0$
		\item[d)] $ind_r(r)\equiv 1[\varphi(n)]$
	\end{itemize}			
\end{thm}
\underline{\textbf{Euler's criterion:}} Let $p$ be an odd prime, and $p \nmid a$. \\Then $a$ is a quadric residue iff:\[a^{\frac{p-1}{2}}\equiv 1[p]\]
\begin{thm} $p$ an odd prime, $p\nmid a$, $p\nmid b$. Then:
	\begin{itemize}
		\item[a)] $a\equiv b[p] \Rightarrow (a/p)=(b/p)$
		\item[b)] $(a^2/p)=1$
		\item[c)] $(a/p)\equiv a^\frac{p-1}{2}[p]$
		\item[d)] $(ab/p)\equiv (a/p)(b/p)$
		\item[e)] $(1/p)=1 \qquad (-1/p)=(-1)^\frac{p-1}{2}$
	\end{itemize}
\end{thm}
\textbf{Corollary:}\[(-1/p)=\left \{ \begin{array}{lcl}
1\\
-1\\
\end{array}
\right. \iff 
\begin{array}{lcl}
p\equiv 1[4]\\
p\equiv 3[4]\\
\end{array}\]
\begin{thm}There are infinitely many primes of the form $4k+1$\end{thm}
\begin{proof}: Assume $p_1 \dots p_n$ are all the primes $\equiv 1[4]$, and set:
	\[ N=(2p_1p_2\dots p_n)^2+1\]
	Let $p$ be a factor in $N$, then $p$ is odd since $N$ is odd.
	\[\begin{array}{lcl}
	p\mid N, \; N\equiv 0[p]\\
	(2p_1p_2\dots p_n)^2+1\equiv 0[p]\\
	\Rightarrow (2p_1\dots p_n)^2\equiv -1[p]\\
	\Rightarrow (-1/p)=1\\
	\Rightarrow p\equiv 1[4]\\
	\Rightarrow p \mid N-(2p_1\dots p_n)=1\text{: absurd!}
	\end{array}
	\]
	Hence there exists infinitely many primes $\equiv 1[4]$.
\end{proof}						
\underline{\textbf{Gauss's lemma:}} $p$ an odd prime, $p\nmid a$\\
Let $n$ denote the number of elements in \[ S=\{a,2a,\dots,({\scriptstyle \frac{p-1}{2}})a\}\] whose remainders [p] lie in $(\frac{p}{2},p)$, then: \[(a/p)=(-1)^n\]
\begin{thm} $p$ an odd prime, then:
	\[(2/p)=\left \{
	\begin{array}{lcl}
	1 \text{ if } p\equiv \pm 1[8]\\
	-1 \text{ if } p\equiv \pm 3[8]\\
	\end{array}
	\right.\]
\end{thm}
\textbf{Corollary:}$(2/p)=(-1)^\frac{p^2-1}{8}$
\begin{thm}$p$ an odd prime.
	\[\sum_{a=1}^{p-1} (a/p)=0\]
	i.e. there are exactly $\frac{p-1}{2}$ quadric residues and $\frac{p-1}{2}$ quadric non residues $[p]$.\\
\end{thm}
\underline{\textbf{Gauss's quadratic reciprocity theorem:}}\\
$p\neq q$ two odd primes, then: \[(p/q)(q/p)=(-1)^{\frac{p-1}{2}\frac{q-1}{2}}\]
The exact proof is in the book.
\textbf{Corollary:} $p\neq q$ two odd primes, then \[(p/q)(q/p)=\left \{ 
\begin{array}{lcl}
1 \text{ if }\: p \:\text{or} \:q \equiv 1[4]\\
-1 \text{ if }\: p \:\text{or} \:q \equiv 3[4]\\
\end{array} 
\right.\]
\textbf{Corollary:} $p\neq q$ two odd primes, then	\[(p/q)=\left\{ 
\begin{array}{lcl}
(q/p) \text{ if }\: p \:\text{or} \:q \equiv 1[4]\\
-(q/p) \text{ if }\: p \:\text{or} \:q \equiv 3[4]\\
\end{array}
\right.\]
\begin{thm} If $p\neq 3$ is an odd prime, then \[ (3/p)=\left\{
	\begin{array}{lcl}
	1 if p\equiv \pm 1 [12]\\
	-1 if p\equiv \pm 5 [12]\\
	\end{array}
	\right.\]
\end{thm}
\underline{\textbf{Lemma:}} \[\text{If  }\left\{
\begin{array}{lcl}
m=a^2+b^2\\
n=c^2+d^2
\end{array}
\right. \Rightarrow mn \;\text{is also a sum of two squares.}\]
\underline{Thue's Lemma:} $p$ a prime, $a\in \mathbb{Z}$, $p\nmid a$\\
Then $ax\equiv y [p]$ has a solution $x_0,y_0 \in \mathbb{Z}$/ 
\[0<|x_0|<\sqrt{p},\qquad 0<|y_0|<\sqrt{p}\]
\underline{\textbf{Fermat's theorem:}} An odd prime $p$ is a sum of 2 squares iff $p\equiv 1[4]$.\\
\textbf{Proposition:} $p$ a prime of the form $4k+1$ can be represented uniquely as a sum of two squares.
\begin{thm} Let $n\in \mathbb{N}$, $n=N^2m$, with m square free.\\
	Then $n=a^2+b^2 \iff m$ contains no prime factor of the form $4k+3$
\end{thm}
\begin{thm} $\gcd(u_n,u_{n+1})=1 \qquad \forall n \geq 1$
\end{thm}
\underline{Proposition:} for $m\geq 2$, $n \geq 1$: \[u_{m+n}=u_{m-1}u_n+u_mu_{n+1}\]
\begin{thm} For $m\geq 1$, $n \geq 1$ \[u_m\mid u_{mn}\]
\end{thm}
\begin{thm} The $\gcd$ of two Fibonacci numbers is a Fibonacci number: \[\gcd(u_m,u_n)=u_{\gcd(m,n)}\]
\end{thm}
\underline{\textbf{Lemma:}} Let $m=qn+r$, with $m,n,q,r \geq 1$, then:
\[\gcd(u_m,u_n)=\gcd(u_r,u_n)\]
\underline{Example:} \[\begin{array}{lcl}
\frac{a}{b} = \frac{43}{13}\\
43=3\cdot 13 +4\\
13=3\cdot 4+1\\
4=4\cdot 1
\end{array}
\vline
\begin{array}{lcl}
\frac{43}{13} = 3+\frac{4}{13}=3+\frac{1}{\frac{13}{4}}\\
\frac{13}{4}=3+\frac{1}{4}=3+\frac{1}{4}{1}\\
\frac{4}{1}=4
\end{array}
\]
\[\Rightarrow \frac{43}{13}=3+\frac{1}{3+\frac{1}{4}}\]
\[\Rightarrow \frac{13}{43}=0+\frac{1}{3+\frac{1}{3+\frac{1}{4}}}\]



Notation: $[a_0;a_1,a_2,\dots,a_n]=a_0+\frac{1}{a_1+\frac{1}{\begin{array}{lcl}
		&\scriptstyle{\vdots}& \qquad \scriptstyle{\vdots \qquad \vdots}\\
		&&\scriptstyle{a_{n-1}+\frac{1}{a_n}}
		\end{array}}}$
\[\frac{13}{43}=[0;3,3,4]\]
\begin{defi} $C_k=[a_0;a_1,a_2,\dots,a_k]_{1\leq k\leq n}$;  is called the $k$\up{th} convergent to $[a_0;a_1,\dots,a_k,\dots,a_n]$.
\end{defi}
\underline{Example:} \[	\left.
\begin{array}{lcl}
[0;3,3,4]\\
C_0=0\\
C_1=0+\frac{1}{3}=\frac{1}{3}\\
C_2=0+\frac{1}{3+\frac{1}{3}}=\frac{3}{10}\\
C_3=0+\frac{1}{3+\frac{1}{3+\frac{1}{4}}}=\frac{13}{43}
\end{array}
\right \} C_k=\frac{p_k}{q_k}\text{, with  }
\left [
\begin{array}{lll}
p_0=a_0&q_0=1\\
p_1=a_1a_0+1&q_1=a_1\\
p_k=a_kp_{k-1}&q_k=a_kq_{k-1}+q_{k-2}\\
\end{array}
\right ]
\]
\begin{thm} The $k$\up{th} convergent $C_k=\frac{p_k}{q_k}$ with $p_k, q_k$ given by recursion.
\end{thm}
\underline{Convention:}
\[\left \{
\begin{array}{lcl}
p_{-2}=0&q_{-2}=1\\
p_{-1}=1&q_{-1}=0
\end{array}
\right.
\left(
\begin{array}{l|l|l|l|l|l|l}
k&-2&-1&0&1&2&3\\ \hline
a_k&.&.&0&3&3&4\\
p_k&0&1&0&1&3&13\\
q_k&1&0&1&3&10&43\\
C_k&.&.&0&\frac{1}{3}&\frac{3}{10}&\frac{13}{43}
\end{array}
\right)
\]
\begin{thm}The convergents $\frac{p_k}{q_k}\qquad \scriptstyle{\forall k\leq n}$ satisfy:\[p_kq_{k-1}-p_{k-1}q_k=(-1)^{k-1}\]
\end{thm}
\underline{\textbf{Lemma:}}  $q_{k-1}\leq q_k$ for $A\leq k \leq n$ with a strict inequality for $k>1$.\\
Hence $q_k\longrightarrow \infty$ as $k\longrightarrow \infty$.
\begin{thm} The convergents satisfy:
	\begin{itemize}
		\item[a)] $C_0<C_2<C_2<\dots$
		\item[b)] $C_1<C_3<C_5<\dots$
		\item[c)] $C_{2s}<C_{2r-1} \quad \forall s\geq 0, r\geq 1.$
	\end{itemize}
\end{thm}
\end{document}
	
