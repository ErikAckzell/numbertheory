\documentclass{report}


\usepackage[utf8]{inputenc}    
\usepackage[english]{babel}
\usepackage[T1]{fontenc}
\usepackage{amsmath}
\usepackage{amsthm}
\usepackage{amssymb}
\usepackage{mathrsfs}     
\usepackage{comment}
\usepackage{setspace}
\usepackage{color}
\usepackage{graphicx}
\usepackage{fancyhdr}
\usepackage{cancel}
\usepackage{mathrsfs}
\usepackage{enumitem}

\newcommand{\Z}{\mathbb{Z}}
\newcommand{\nequiv}{\cancel{\equiv}}
\newcommand{\up}[1]{\textsuperscript{#1}}
\DeclareMathOperator{\ord}{ord}

\title{Course: Number Theory}
\begin{document}
\newtheorem*{defi}{Definition}
\newtheorem*{thm}{Theorem}

{\centering
\Large
\textbf{Number Theory}\\
\normalsize
\textbf{November 4}\\
2014\\
}
\vspace{5mm}
\subsection*{Congruences and properties of Congruences}
Let $a,n \in \mathbb{Z},$ then $n$ divides $a$ if $\exists b/\; a=nb$,$\qquad b \in \mathbb{Z}$.\\
$n$ is a divisor of $a$.
\begin{defi} $p \in \mathbb{Z}$ is a \textbf{prime number} if $p > 1$ and $\pm1$, $\pm p$ are the only divisors of $p$.
\end{defi}
\begin{defi} $a,b,n \in \mathbb{Z}, \; n\geq 1$. \\
$a$ is \textbf{congruent to} $\textbf{b}$ \textbf{modulo} $\textbf{n} \iff n \mid (a-b) \iff a-b=nk$.\\
We then write: $a\equiv b \pmod n$ or $a \equiv b [n]$.
\end{defi}
\underline{Example:}
\[
-31 = -42+11=(-6)7 +11\equiv 11[7]
\]
\[
a=nq+r,\quad 0 \leq r<n \quad \Rightarrow \quad a \equiv r [n]
\]
r is called the remainder of a divided by n.

\begin{thm} $a,b \in \mathbb{Z}$, then $a \equiv b[n] \iff a$ and $b$ have the same remainder when divided by n.
\end{thm}
\begin{proof}
\underline{$\Rightarrow$}
\[
\left.
\begin{array}{r r r}
a \equiv b[n]\\
a=kn+b\\
b=qn+r\\
\end{array}
\right\}
\rightarrow a=qn+r+kn=(q+k)n+r
\]
\underline{$\Leftarrow$}
\[
\left.
\begin{array}{r r r}
a=q_1n+r\\
b=q_2n+r\\
\end{array}
\right \} \Rightarrow a-b=(q_1-q_2)n\equiv0[n] \Rightarrow a\equiv b[n]
\]
\end{proof}

\begin{thm} $n>1,$  $a,b,c,d \in \mathbb{Z}$
\begin{enumerate}
\item $a \equiv a[n]$
\item $a \equiv b[n] \Rightarrow b \equiv a [n]$
\item If  $a \equiv b[n], \; b \equiv c[n] \text{then } a\equiv c [n]$
\item If  $a \equiv b[n], \; c \equiv d[n]$ then $
	\left \{
	\begin{array}{l l l}
	a+c \equiv b+d [n]\\
	ac \equiv bd [n]
	\end{array}
	\right.
$
\item $a \equiv b[n] \Rightarrow 
	\left \{
	\begin{array}{l l l}
	a+c \equiv b+c[n]\\
	ac\equiv bc[n]
	\end{array}
	\right.
$
\item $a \equiv b \Rightarrow a^k \equiv b^k [n]$
\end{enumerate} 
\end{thm}
%\begin{proof}
%\begin{enumerate}
%\item by reflexivity
%\item by symmetry
%\item by transitivity
%\item If $a-b=kn$ and $c-d=k'n$ then 
%\end{enumerate}
%\end{proof}
\underline{Example:}
Let's see if $41 \mid 2^{20}-1$.
\[
\begin{array}{l l l}
2^{20}=(2^5)^4\\\
2^5=32 \equiv -9[41]\\
2^{20} \equiv -9^4[41] \equiv 81^2 \equiv (-1)^2[41] \equiv 1[41]\\
\Rightarrow 2^{20} \equiv 1[41]\\
\Rightarrow 2^{20}-1\equiv 0[41]\\
\end{array}
\]


\[
\begin{array}{r c l}
\text{{\fontencoding{U}\fontfamily{futs}\selectfont\char 66\relax}} \qquad 2 \cdot 4 &\equiv& 2 \cdot 1[6]\\
\cancel{\Rightarrow} 4 &\equiv& 1[6]
\end{array}
\]
\begin{thm}
\[
ca\equiv cb[n] \Rightarrow a\equiv b [ \frac{n}{d}] \qquad \text{where } d=\gcd(n,c)
\]
\end {thm}
\begin{proof}
\[
c(a-b)=ca-cb=kn \qquad k\in \mathbb{Z}
\]
As 
\[
d=gcn(n,c) \Rightarrow
\left.
\begin{array}{r r r}
n&=&dr\\
c&=&ds\\
\end{array}
\right \}
\gcd(r;s)=1
ds(a-b)=kdr
\]
\[
s(a-b)=dr \Rightarrow r \mid s(a-b)
\]
But
\[
\gcd(r,s)=1 ( \Rightarrow r \cancel{\mid} s)
\]
\[
\Rightarrow r \mid (a-b) \Rightarrow a \equiv b[ \frac{n}{b} ]
\]
\end{proof}
\begin{enumerate}
  \item \underline{Corollary:} If $ca \equiv cb [n]$ and $\gcd(n,c)=1$, then $a \equiv b[n]$.
  \item \underline{Corollary:} If $ca \equiv cb [p]$, $p$ prime and $p \cancel{\mid} c$, then $a \equiv b[p]$.
\end{enumerate}



%\large \textbf{Chinese remainder theorem}: \small (Sun-Tse) \normalsize
\subsection*{Chinese remainder theorem}
\[ 
\text{Let  }
\left.
\begin{array}{r c l}
n_1 \dots n_r \in \mathbb{N}\\
a_1 \dots a_r \in \mathbb{Z}\\
\end{array}
\right/ \gcd(n_i,n_j)=1 \forall i \neq j
\]
then the  system
$
\left \{
\begin{array}{r c l}
x &\equiv& a_1[n_1]\\
x &\equiv& a_2[n_2]\\
\vdots &\vdots& \vdots\\
x &\equiv& a_r[n_r]\\
\end{array}
\right.
$ \\
has a simultaneous solution which is unique modulo $n_1 \cdot n_2 \dots n_r$
\[
\Rightarrow f(x) \equiv 0[n] \text{ with  } n = p_1^{k_1} \dots p_r^{k_r} \qquad 
\left \{
\begin{array}{r c l}
f(x) &\equiv& 0[p_1^{k_1}]\\
&\vdots&\\
f(x) &\equiv& 0[p_r^{k_r}]\\
\end{array}
\right.
\]
\newpage
\begin{proof}
Set 
\[
n=n_1 \dots n_r
\]
And
\[
N_k=\frac{n}{n_k}=n_1 n_2 \dots n_{k-1} n_{k+1} \dots n_r
\]
Then, $N_k x_k \equiv 1 [n_k]$\\
\[
N_k x_k + n_k y_k = 1, \text{exists because } \gcd(N_k, n_k)=1
\]
And if we set $\bar{x} \equiv \sum_{k=1}^{r}a_k N_k x_k \equiv a_k [n_k]$, then $\bar{x}$ is a simultaneous solution.\\



\underline{Uniqueness:} \\
Suppose $\bar{x}'$ is another solution, then for $1 \leq k \leq r$,
\[
\bar{x} \equiv a_k \equiv \bar{x}' [n_k]
\]
\[
n_k \mid (\bar{x} - \bar{x}')
\]
\[
\Rightarrow n_1 \dots n_r \mid (\bar{x}-\bar{x}')
\]
Hence $\bar{x}' \equiv \bar{x} [n_1 \dots n_r]$.
\end{proof}


\underline{Example:}
\[
\left.
\begin{array}{r c l}
x \equiv 2 [3]\\
x \equiv 3 [5]\\
x \equiv 2 [7]\\
\end{array}
\right \}
\]
\[ n=3 \cdot 5 \cdot 7 = 105 \]
\[N_1=\frac{105}{3}=35\]
\[N_2=\frac{105}{5}=21\]
\[N_3=\frac{105}{7}=15\]
That gives us the following system:
\[
\left.
\begin{array}{r c l}
35x_1=N_1 x_1 \equiv 1[3]\\
21x_2=N_2 x_2 \equiv 1[5]\\
15x_3=N_3 x_3 \equiv 1[7]\\
\end{array}
\right \}
\Rightarrow
\left .
\begin{array}{r c l}
2x_1 \equiv 1[3]\\
x_2 \equiv 1 [5]\\
x_3 \equiv 1 [7]\\
\end{array}
\right \}
\Rightarrow
\left.
\begin{array}{r c l}
x_1=2\\
x_2=1\\
x_3=1\\
\end{array}
\right \}
\]
\[ 
\begin{array}{r c l}
\Rightarrow \bar{x}&=&a_1N_1x_1+a_2N_2x_2+a_3N_3x_3\\
\bar{x}&=& 2 \cdot 35 \cdot 2 + 3 \cdot 21 \cdot 1 + 2 \cdot 15 \cdot 1 = 233\\
\end{array}
\]
\newpage
\subsection*{Fermat's theorem}
Let $p$ be a prime, $a \in \mathbb{Z} / \quad p \cancel{\mid} a$. Then,
\[ a^{p-1} \equiv 1 [p] \]


\begin{proof}
We claim that the elements in the set 
\begin{equation*}
S = \{ka \mid k = 1, 2, \dots, p-1\}
\end{equation*}
are all mutually incongruent modulo $p$. Assume that $1\leq k_1 < k_2 \leq p-1$ and that
\begin{equation*}
k_1a\equiv k_2a[p]
\end{equation*}
As $\gcd(a, p) = 1$, we can cancel $a$ from both sides of the equivalence, obtaining
\begin{equation*}
k_1\equiv k_2 [p],
\end{equation*}
contradicting $1\leq k_1 < k_2 \leq p-1$. Hence, our claim is valid.\\
As $S$ contains exactly $p-1$ elements, each one is congruent to exactly one of $\{1, 2,\dots, p-1\}$ in some order. Hence, we have that
\begin{equation*}
a\cdot 2a\cdots (p-1)a \equiv 1\cdot 2\cdots (p-1)[p]
\end{equation*}
and after some rearranging of the factors in the left hand side, we obtain
\begin{equation*}
a^{p-1}(p-1)!\equiv (p-1)![p].
\end{equation*}
As $\gcd(p, (p-1)!)=1$, we can cancel $(p-1)!$ from both sides of the equivalence, obtaining
\begin{equation*}
a^{p-1}\equiv 1[p]
\end{equation*}
\end{proof}
\underline{Corollary:} If $p$ is prime, and $a \in \mathbb{Z}$, then \[ a^p \equiv a [p] \]
\begin{defi} $n$ is \textbf{composite} if $n>1$ and $n=ab$ with $a,b \in (1,n)$.
\end{defi}
\begin{defi} If for some $a \in \mathbb{Z}$, $a^n \cancel{\equiv}a [n]$, then $n$ is not prime.\\
n is called an absolute pseudoprime if n is composite and \[a^{n-1} \equiv 1[n], \quad \forall a\in \{k\in\mathbb{Z}\mid\gcd(k,n)=1 \}\]
\end{defi}
\newpage

{\centering
\Large
\textbf{Number Theory}\\
\normalsize
\textbf{November 7}\\
2014\\
}
\vspace{10mm}
\subsection*{Wilson's theorem}
If $p$ is a prime, then $(p-1)! \equiv -1[p]$
\begin{proof}
Let $a \in \{ 1,2,\dots,(p-1) \}$. Then $\exists ! a' \in \{ 1,2,\dots,(p-1) \}$ such that
\[ aa' \equiv 1 [p] \]
If $a=a'$ then $a=1$ or $a=p-1$ because:
\[
\begin{array}{r c l}
a^2 \equiv 1[p] &\iff& p \mid (a^2-1)\\
&\Rightarrow& p \mid (a+1)(a-1)\\
&\Rightarrow& p \mid (a+1) \text{ or } p \mid (a-1)\\
&\Rightarrow& a=1 \text{ or } a=p-1 \\
\end{array}
\]
If we group all the elements remaining from $ \{ 2 \dots p-2 \} $ into $\frac{p-3}{2}$ pairs equal to $1[p]$:
\[
\begin{array}{r c l}
&(p-2)!&=2 \cdot 3 \cdots (p-2) \equiv 1[p]\\
&(p-1)!& \equiv -1[p]\\
\end{array}
\]
\end{proof}
\underline{Example:} $p=11$
\[
\left.
\begin{array}{r c l}
2 \cdot 6 \equiv 1[11]\\
3 \cdot 4 \equiv 1[11]\\
5 \cdot 9 \equiv 1[11]\\
7 \cdot 8 \equiv 1[11]\\
\end{array}
\right \}
2 \cdot 6 \cdot 3 \cdot 4 \cdot 5 \cdot 9 \cdot 7 \cdot 8 = 9! \equiv 1[11] \]
And $10! \equiv 10 \equiv -1[11]$.
%\vspace{5mm}
\begin{thm} Let $p$ be an odd prime. \\Then $x^2+1 \equiv 0 [p]$ has a solution $\iff p \equiv 1[4]$.
\end{thm}
%	\newpage
\begin{proof}
\underline{$\Rightarrow$}\\
Let $a$ be a solution of $a^2 \equiv -1[p]$. We first note that $p\cancel{\mid}a$. Raising both sides of the equivalence relation to the power $\frac{p-1}{2}$, we obtain
\[
\begin{array}{r c l}
&(a^2)^{\frac{p-1}{2}}& \equiv (-1)^{\frac{p-1}{2}}[p]\\
\Rightarrow &a^{p-1}&\equiv (-1)^{\frac{p-1}{2}}[p]
\end{array}
\]
By Fermat's theorem, we have that \[a^{p-1}\equiv 1 [p]\] which implies \[2\mid \frac{p-1}{2}\quad \Rightarrow \quad 4\mid p-1\quad \Rightarrow \quad p\equiv 1[4].\]
\underline{$\Leftarrow$}\\
\[
\begin{array}{l c l}
p \equiv 1[4]\\
(p-1)!=1\cdot 2\cdots (p-1)\\
\end{array}
\]
The factors can be rearranged in the following way:
\begin{equation*}
\begin{aligned}
(p-1)! &= 1\cdot 2\cdots (p-2)(p-1)\\
&\equiv 1\cdot 2\cdots \left(\frac{p-1}{2}\right)\left(-\frac{p-1}{2}\right)\cdots (-2)(-1)\\
&=(-1)^{\frac{p-1}{2}}\left[\left(\frac{p-1}{2}\right)!\right]^2[p].
\end{aligned}
\end{equation*}
By Wilson's theorem, \begin{equation*}
-1\equiv(-1)^{\frac{p-1}{2}}\left[\left(\frac{p-1}{2}\right)!\right]^2[p] 
\end{equation*}
and as $4\mid p-1$, $p=4k + 1$ for some $k\in \mathbb{Z}$, and so \begin{equation*}
\begin{aligned}
(-1)^{\frac{p-1}{2}}&=(-1)^{\frac{4k + 1-1}{2}}\\
&=1.
\end{aligned}
\end{equation*}
Hence, 
\begin{equation*}
-1\equiv\left[\left(\frac{p-1}{2}\right)!\right]^2[p]
\quad\iff\quad
\left[\left(\frac{p-1}{2}\right)!\right]^2+1\equiv 0[p],
\end{equation*}
so
\begin{equation*}
x=\left(\frac{p-1}{2}\right)!
\end{equation*}
is a solution to the equation.
\end{proof}
\underline{Example:} 
\[
\begin{array}{l c l}
p=13 \equiv 1[4]\\
\text{Set } a=\frac{p-1}{2}!=6!=720 \equiv 5 [13]\\
\Rightarrow 720^2+1 \equiv 5^2+1 \equiv 26 \equiv 0[13]\\
\end{array}
\]
\subsection*{Number Theoretic Functions}
\begin{defi}
	A function $f$ is said to be \textbf{number theoretic} if its domain of definition is $\mathbb{Z}^+$.
\end{defi}
The number theoretic functions which we will use the most are:
\[
\left \{
\begin{array}{l c l}
&\tau(n)& = \sum_{d \mid n}1 \text{,   the number of positive divisors of n.}\\
&\sigma(n)&= \sum_{d \mid n}d \text{,   the sum of the positive divisors of n.}\\
\end{array}
\right.
\]
\underline{Example:} For $n=10$
\[
\begin{array}{l c l}
\tau(10)=4\\
\sigma(10)=1+2+5+10=18\\
\end{array}
\]
\textbf{Observation:} Let $n>1$ with  $n=p_1^{k_1} \cdots p_n^{k_n}$ where each $p_i$ is a distinct prime.\\
Then the positive divisors of n are exactly
\[d=p_1^{a_1} \cdots p_r^{a_r},\quad 0 \leq a_i \leq k_i, \quad 1\leq i\leq r. \]
\begin{thm} Let $n>1$ with $n=p_1^{k_1} \cdots p_r^{k_r}$. Then \\
	\begin{enumerate}
		\item $\tau(n)=(k_1+1) \cdots (k_r+1)$
		\item $\sigma(n)=(\frac{p_1^{k_1+1}-1}{p_1-1})\dots(\frac{p_r^{k_r+1}-1}{p_r-1})$
	\end{enumerate}
\end{thm}
\begin{proof}
	\begin{enumerate}
		\item Each $a_i$ in $d=p_1^{a_1} \dots p_r^{a_r}$ can be chosen in $(k_i+1)$ ways. So $a_1 \dots a_r$ can be chosen in $(k_r+1) \cdots (k_1+1)$ ways.
		\item \[
						\begin{array}{l c l}
						&&(1+p_1+p_1^2+\dots+p_r^{k_1})(1+p_2+\dots+p_r^{k_2})\dots(1+p_r+\dots+p_r^{k_r})\\
						&=& (\frac{p_1^{k_1+1}-1}{p_1-1})\dots(\frac{p_r^{k_r+1}-1}{p_r-1})\\
						\end{array}
					\]
	\end{enumerate}
\end{proof}
In general we have 
	\begin{itemize}
		\item $\tau(mn) \neq \tau(m) \tau(n)$
		\item $\sigma(mn) \neq \sigma(m) \sigma(n)$
	\end{itemize}
\begin{defi}
	A number theoretic function $f$ is said to be \textbf{multiplicative} if
		\[\gcd(m,n)=1 \Rightarrow f(mn)=f(m)f(n) \]
\end{defi}
\underline{Lemma :} If $\gcd(m,n)=1$ then the divisors of $mn$ are:
	\[\mathcal{D} = \{d_1 d_2 : d_1 \mid m_1 \quad \text{and}\quad d_2 \mid m_2 \} \]
\begin{proof}
	Let
	\[
	\begin{array}{l c l}
		m=p_1^{k_1} \dots p_r^{k_r}\\
		n=q_1^{j_1} \dots q_s^{j_s}\\
		\gcd(m,n)=1\\
	\end{array}
	\]
	Then $\forall i,j$, $q_i \neq p_i$, so if $d\mid mn$, then
		\[
			\begin{array}{l l l}
				&d=&\underbrace{p_1^{a_1} \dots p_r^{a_r}}_{d_1} \underbrace{q_1^{b_1} \dots q_s^{b_s}}_{d_2}
			\end{array}
		\]
	where $d_1\mid m_1$ and $d_2\mid m_2$. Thus, $d\in \mathcal{D}$.
\end{proof}
\begin{thm} Let f be a multiplicative number theoretic function and define $F(n)$ by:
			\[F(n)= \sum_{d\mid n} f(d), \qquad n \geq 1 \]
			Then $F$ is also a multiplicative number theoretic function.
\end{thm}
\begin{proof}
	Assume $\gcd(m,n)=1$. Then
		\[ F(m,n)=\sum_{d \mid mn} f(d) = 
		\sum_{\substack{d_1 \mid m\\ d_2 \mid n}} f(d_1d_2)= \sum_{\substack{d_1 \mid m\\ d_2 \mid n}} f(d_1)f(d_2)=\sum_{d_1 \mid m} f(d_1) \sum_{d_2 \mid n} f(d_2)
		\]
	$=F(m)F(n)$
\end{proof}
\underline{Corollary :} $\sigma$ and $\tau$ are multiplicative:\\
		\[
			\begin{array}{r c l}
				\text{Write }&\tau(n)& =\sum_{d\mid n} 1 = \sum_{d \mid n} f(d) \qquad \text{ with }f(d)=1\\
				&\sigma(n)&= \sum_{d \mid n} d = \sum_{d\mid n} g(d) \qquad \; \text{with } g(d)=d\\
			\end{array}
		\]
\newpage 
		
{\centering
\Large
\textbf{Number Theory}\\
\normalsize
\textbf{November 11}\\
2014\\
}
\vspace{5mm}
\begin{defi} The M\"{o}bius function $\mu$ is defined by:
	\[ \mu(n)= 	\left \{
						\begin{array}{lcl}
							1 \quad \qquad\text{ if } n=1\\
							0 \quad \qquad \text{ if } \exists p \text{ prime such that } p^2 \mid n\\
							(-1)^r \quad \text{ if }n=p_1p_2 \dots p_r \text{,  with distinct primes.}
						\end{array}
						\right.
	\]
\end{defi}
\underline{Example:}\\
\[ \begin{array}{l c l}
			\mu(6)=(-1)^2 \qquad 6=2\cdot 3\\
			\mu(12)=0 \quad \qquad 12=2^2\cdot 3\\
		\end{array}
\]
\begin{thm}
		$\mu$ is multiplicative.
\end{thm}
\begin{proof}
				Let $\gcd(m,n)=1$ and assume $p^2 \mid m$ or $p^2\mid n$, with $p$ prime. Without loss of generality, say $p^2 \mid m$. Then \begin{equation*}
				\begin{aligned}
				\mu(mn) &=\mu(p^2q)\\
				&=0\\
				&=0\cdot \mu(n)\\
				&=\mu(m)\mu(n)
				\end{aligned}
				\end{equation*}
				and we are done.
				Now let $m>1$ and $n>1$ both be square free:
					\[\begin{array}{l c l}
						m=p_1p_2 \dots p_r\\
						n=q_1q_2 \dots q_s
						\end{array}
					\]
					with $p_i$, $q_j$ distinct primes.
				Then:
					\[
						\mu(m,n)=(-1)^{r+s}=(-1)^r(-1)^s=\mu(m) \mu(n)
					\]
				The cases when exactly one or both of $m$ and $n$ equals 1 are left to the reader.
\end{proof}
\begin{thm}
			For $n\geq1$,
				\[ \sum_{d \mid n} \mu(d) = \left \{
																		\begin{array}{lcl}
																			1 \quad n=1\\
																			0 \quad n>1\\
																		\end{array}
																		\right.
				\]
\end{thm}
\begin{proof}
				Set $F(n)=\sum_{d \mid n} \mu(d)$. As $\mu$ is multiplicative, so is $F$.\\
				For $n=1$ we have \[ F(1)= \mu(1)=1\]
				Assume $n>1$, $n=p_1^{k_1} \dots p_r^{k_r}$, with $p_i$ distinct primes. For any prime $p$, we have that \begin{equation*}
				\begin{aligned}
				F(p^k)&=\sum_{d \mid p^k} \mu(d)\\
				&= \mu(1) + \mu(p) + \mu(p^2) + \dots + \mu(p^k)\\
				&= 1 -1 + 0 + 0 + \dots + 0\\
				&=0
				\end{aligned}
				\end{equation*}
				and by multiplicity, \begin{equation*}
				\begin{aligned}
				F(n)&=F(p_1^{k_1}) \dots F(p_r^{k_r})\\
				&=0.
				\end{aligned}
				\end{equation*}
\end{proof}
\subsection*{M\"{o}bius inversion formula}
Let $F$ and $f$ be connected by \[F(n)=\sum_{d \mid n} f(d).\]
Then 
\[ f(n)=\sum_{d\mid n} \mu(d) F(\frac{n}{d}) = \sum_{d \mid n} \mu(\frac{n}{d}) F(d)\]
\begin{proof}
	We first show that the two sums are indeed equal. Set $d'=\frac{n}{d}$. As $d$ ranges over all the divisors of $n$, so does $d'$. Thus, 
	\begin{equation*}
	\begin{aligned}
	\sum_{d\mid n} \mu(d) F(\frac{n}{d})&=\sum_{d\mid n} \mu(\frac{n}{d'}) F(d')\\
	&=\sum_{d'\mid n} \mu(\frac{n}{d'}) F(d')
	\end{aligned}
	\end{equation*} 
	Furthermore, we have
	\begin{equation}\label{mobiussums}
	\begin{aligned}
	\sum_{d\mid n} \mu(d) F(\frac{n}{d}) &= \sum_{d\mid n}\left(\mu(d)\sum_{c\mid \frac{n}{d}}f(c) \right)\\
	&= \sum_{d\mid n}\left(\sum_{c\mid \frac{n}{d}}\mu(d)f(c) \right)
	\end{aligned}
	\end{equation}
	We note that
	\begin{equation*}
	d\mid n \wedge c\mid\frac{n}{d} \quad \Leftrightarrow \quad c\mid n \wedge d\mid \frac{n}{c}
	\end{equation*}
	Thus, we can rewrite \eqref{mobiussums} as follows
	\begin{equation}\label{mobiussums2}
	\begin{aligned}
	\sum_{d\mid n}\left(\sum_{c\mid \frac{n}{d}}\mu(d)f(c) \right)&= \sum_{c\mid n}\left(\sum_{d\mid \frac{n}{c}}\mu(d)f(c) \right)\\
	&= \sum_{c\mid n}\left(f(c)\sum_{d\mid \frac{n}{c}}\mu(d) \right)
	\end{aligned}
	\end{equation}
	As by our previous theorem,
				\[ \sum_{d \mid n} \mu(d) = \left \{
				\begin{array}{lcl}
				1 \quad n=1\\
				0 \quad n>1\\
				\end{array}
				\right.
				\]
	the last sum in \eqref{mobiussums2} reduces to 
	\begin{equation*}
	\begin{aligned}
	\sum_{c=n}\left(f(c)\sum_{d\mid \frac{n}{c}}\mu(d) \right) &= \sum_{c=n}f(c)\cdot 1\\
	&=f(n)
	\end{aligned}
	\end{equation*}	
\end{proof}
\underline{Example:} \\
$\sigma(n) = \sum_{c \mid n} d ; \qquad f(n)=n$\\
M\"{o}bius inverse: \[ n= \sum_{d \mid n} \mu(\frac{n}{d}) \sigma(d)\]
\begin{thm}
				Let $f$, $F$ be connected by:
					\[F(n)=\sum_{d \mid n} f(d) ;\qquad \text{ if }F\text{is multiplicative, then }f\text{ is multiplicative}\]
\end{thm}
\begin{proof}
		Let $\gcd(m,n)=1$. Then 
\begin{equation*}
\begin{aligned}
f(mn)&=\sum_{d \mid mn} \mu(d) F(\frac{mn}{d})\\
&=\sum_{\substack{d_1 \mid m\\d_2 \mid n}} \mu(d_1 d_2) F(\frac{m}{d_1} \frac{n}{d_2})\\
&=\sum_{\substack{d_1 \mid m \\d_2 \mid n}} \mu(d_1) \mu(d_2) F(\frac{m}{d_1})F(\frac{n}{d_2})\\
&=\left(\sum_{d_1 \mid m} \mu(d_1) F(\frac{m}{d_1})\right) \left(\sum_{d_2 \mid n} \mu(d_2) F(\frac{n}{d_2})\right)\\
&=f(n)f(m)		
\end{aligned}
\end{equation*}
													
																										
													
\end{proof}
\subsection*{Euler's $\varphi$ function} 
For $n\geq 1$ Euler's $\varphi$ function is defined as $\varphi(n)=$ the number of integers in $\{1,2,\dots,n\}$ that are relatively prime to $n$. In other words,
													\[\varphi(n)=\mid \{ a \in \mathbb{Z} :1 \leq a \leq n,\quad \gcd(a,n)=1 \} \mid \]
\underline{Example:} Let $n = 18$. Then $\{ a \in \mathbb{Z} :1 \leq a \leq n,\quad \gcd(a,n)=1 \} = \{1,5,7,11,13,17\}$, so $\varphi(18)=6$\\
\begin{thm}
				For $p$ prime, $k>0$:
															\[\varphi(p^k)=p^k-p^{k-1}=p^k(1-\frac{1}{p})\]
\end{thm}
\begin{proof}
				\[\gcd(n,p^k)=1 \iff p \cancel{\mid} n\]
				The multiples of $p$ in $[1, p^k]$ are $p, 2p, \dots,pp, \dots, p^{k-1}p$, adding up to $p^{k-1}$ numbers.\\
				Hence \[\varphi(p^k) =p^k-p^{k-1}=p^k(1-\frac{1}{p})\]
\end{proof}
\underline{Lemma :} For $a,b,c \in \mathbb{Z},\; \gcd(a,bc)=1 \iff \gcd(a,b)=1=\gcd(a,c)$.
\begin{proof}
		\underline{$\Rightarrow$} Trivial.\\
		\underline{$\Leftarrow$} Each prime factor in $a$ is distinct from every prime factor in $b$ and in $c$. Hence $\gcd(a,bc)=1$.
\end{proof}
\begin{thm} $\varphi$ is multiplicative.
\begin{proof}
Let $m, n\in \Z$ with $\gcd(m, n) = 1$. Since $\varphi(1)=1$, the result holds when $m=1$ or $n=1$.\\
Let $m>1, n>1$ and construct a table consisting of all numbers $1, \dots, mn$ in the following way
\begin{equation}\label{eulermatrix}
\begin{array}{rrrrrr}
1 & 2 & \cdots & r & \cdots & m\\
m + 1 & m + 2 & & m + r & & 2m\\
2m + 1 & 2m + 2 & & 2m + r & & 3m\\
\vdots & \vdots & & \vdots && \vdots\\
(n-1)m + 1 & (n-1)m + 2 & & (n-1)m + r & &nm
\end{array}
\end{equation}
We know that $\varphi(mn)$ is equal to the number of entries in the table which are relatively prime to $mn$ and by our previous lemma, this is the same as the number of entries which are relatively prime to both $m$ and $n$.\\
We note that $\gcd(qm + r, m) = \gcd(r, m)$, so the numbers in the $r$th column are relatively prime to $m$ if and only if $\gcd(r, m) = 1$. Hence, there are $\varphi(m)$ columns of numbers relatively prime to $m$. If we can show that there are $\varphi(n)$ numbers in each such column which are relatively prime to $n$ we are done.\\
Assume that $\gcd(r, m)=1$. Consider the set of $n$ integers in the $r$th column
\begin{equation*}
R = \{km + r : k=0, \dots (n-1)\}
\end{equation*}
We claim that these are pairwise incongruent modulo $n$. Assume that $0\leq k_1 < k_2 \leq n-1$ and that 
\begin{equation*}
k_1m + r\equiv k_2m + r[n]
\end{equation*}
Subtracting $r$ and canceling $m$ from both sides of the equation results in 
\begin{equation*}
k_1\equiv k_2[n],
\end{equation*}
which is a contradiction. Thus, our claim holds and the numbers in $R$ are congruent modulo $n$ to $0, 1, \dots, n-1$ in some order.\\
We note that if $s\equiv t[n]$ and $\gcd(s, n)=1$, then $\gcd(t, n)=1$. Hence, the number of integers in $R$ relatively prime to $n$ are $\varphi(n)$, which is what we wanted to show.
\end{proof}
\end{thm}
%%%%%%%%%%% ADD PROOF HERE %%%%%%%%%%%%%%%%
\begin{thm}
				For $n=p_1^{k_1}\dots p_r^{k_r} ;\quad \varphi(n)=n(1-\frac{1}{p_1})\dots(1-\frac{1}{p_r})$
\end{thm}
\begin{proof}
					\[ 	\begin{array}{l c l}
							&\varphi(n)&=\varphi(p_1^{k_1})\dots\varphi(p_r^{k_r})\\
							&&=p_1^{k_1}(1-\frac{1}{p_1})\dots p_r^{k_r}(1-\frac{1}{p_r})\\
							&&=n(1-\frac{1}{p_1})\dots (1-\frac{1}{p_r})\\
							\end{array}
					\]
\end{proof}
\underline{Example :} $\varphi(18)=\varphi(2 \cdot 3^2)=18(1-\frac{1}{2})(1-\frac{1}{3}) = 18 \frac{1}{2} \frac{2}{3}=6$\\
\newpage
{\centering
\Large
\textbf{Number Theory}\\
\normalsize
\textbf{November xx}\\
2014\\
}
\vspace{10mm}
\underline{Lemma :} Let $n>1$, and $\gcd(a, n)=1$. If 
\begin{equation*}
a_1, \dots, a_{\varphi(n)}\in [1, n)
\end{equation*} 
are the positive integers which are relatively prime to $n$, then 
\begin{equation*}
aa_1, \dots, aa_{\varphi(n)}
\end{equation*}
are congruent to $a_1, \dots, a_{\varphi(n)}$ modulo $n$ in some order.
\begin{proof}
	We claim that the $aa_i$s are pairwise incongruent modulo $n$. Assume that
\begin{equation*}
aa_i\equiv aa_j[n], \quad i<j.
\end{equation*}
Since $\gcd(a, n) = 1$, $a_i\equiv a_j[n] \Rightarrow a_i=a_j \Rightarrow i=j$.\\
As $\gcd(a, n)=1$ and $\gcd(a_i, n) = 1$, we have that $\gcd(aa_i, n)=1$. Let 
\begin{equation*}
aa_i\equiv b[n].
\end{equation*}
Then 
\begin{equation*}
1=\gcd(aa_i, n) = \gcd(b, n),,
\end{equation*}
so $b=a_j$ for some j.
\end{proof}
\subsection*{Euler's theorem}
Let $n>1$ and $\gcd(a, n)=1$. Then 
\begin{equation*}
a^{\varphi(n)}\equiv 1 [n]
\end{equation*}
\begin{proof}
	Let $a_1, \dots, a_{\varphi(n)}$ be as in our previous lemma and write
	\begin{equation*}
	\left\{\begin{array}{lcl}
	aa_1 & \equiv & a'_1 [n]\\
	&\vdots\\
	aa_{\varphi(n)} & \equiv & a'_{\varphi(n)} [n]\\
	\end{array}\right.
	\end{equation*}
	so that after permutation, 
	\begin{equation*}
	\{a'_1, \dots, a'_{\varphi(n)}\} = \{a_1, \dots, a_{\varphi(n)}\}.
	\end{equation*}
We have that 
\begin{equation*}
(aa_1)\cdots(aa_{\varphi(n)})\equiv a'_1\cdots a'_{\varphi(n)}=a_1\cdots a_{\varphi(n)}[n]
\end{equation*}
implying that 
\begin{equation*}
a^{\varphi(n)}(a_1\cdots a_{\varphi(n)})\equiv 1\cdot (a_1\cdots a_{\varphi(n)}) [n].
\end{equation*}
As for each $i$, $\gcd(a_i, n)=1$, $\gcd(a_1\cdots a_{\varphi(n)}, n)=1$. Thus we can cancel the factors and obtain
\begin{equation*}
a^{\varphi(n)}\equiv 1[n],
\end{equation*}
which is what we wanted to show.
\end{proof}
\subsection*{Gauss' theorem}
\newpage
{\centering
	\Large
	\textbf{Number Theory}\\
	\normalsize
	\textbf{November 18}\\
	2014\\
}
\vspace{10mm}
\begin{defi} The \textbf{order} of $a[n]$ is the smallest positive integer $k$ such that
																\[ a^k \equiv 1 [n] \]
						We write $k= \ord_n(a)$
\end{defi}
\underline{Example:} $a=2$, $n=7$\\
	$2^3=8 \equiv 1 [7]$\\
	$2^1 \equiv 2, \; 2^2=4\equiv 1 [7]$\\

\begin{thm}	Let $\ord_n(a)=k$, then:
				\[a^n\equiv 1 [n] \iff k\mid n \]
				In particular, $k\mid \varphi(n)$
\end{thm}
\begin{proof}
\underline{$\Leftarrow$}  \[ n=qk,\quad q \in \mathbb{Z} \]
													\[a^n=(a^k)^q\equiv 1^q \equiv 1 [n] \]
\underline{$\Rightarrow$} \[ n=qk+r, \qquad 0 \leq r <k \]
													\[ 1 \equiv a^n \equiv (a^k)^q a^r \equiv a^r [n] \]
Now if $r=0$, $n=qk$, then $k \mid n$ and we are done. If $r\neq 0$, then this contradicts $\ord_n(a)=k$, as $0<r<k$ and the result follows.\\
Lastly, Euler's theorem implies $a^{\varphi(n)}\equiv 1[n]$, and so $k \mid \varphi(n)$.
\end{proof}
\underline{Example:} $k=\ord_{11}(2) \qquad 2^{10}\equiv 1[11]$\\
\[ k \mid 10 \Rightarrow k \in \{1,2,5,10\}\]
If $k=$ 
				\begin{description}[labelindent=1cm]
				\item[$1:$] $2^1=2 \nequiv 1[11]$
				\item[$2:$] $2^2=4 \nequiv 1[11]$
				\item[$5:$] $2^5=32 \equiv -1 \nequiv 1[11]$
				\end{description}
				Hence $\ord_{11}(2)=10$
\begin{thm} Let $\ord_n(a)=k$, then:
						\[\begin{array}{lcl}
							a^i\equiv a^j[n]\\
							\iff i=j[k]\\
							\end{array}
						\]
\end{thm}
\begin{proof} \underline{$\Rightarrow$}\\
							Say $i \leq j$\\
							\[a^i\equiv a^j \equiv a^{i+(j-i)}\equiv a^i a^{j-i}[n]\]
							\[\gcd(a^i,n)=1,\; 1\equiv a^{j-i}[n]\]
							The precedent theorem says: \[k \mid j-i \Rightarrow j \equiv i[k]\]
							
\underline{$\Leftarrow$}\\
							\[j=qk+i\]
							\[a^j=a^{qk+i}=\underbrace{(a^k)^q}_{\equiv1} a^i\equiv a^i[n]\]
\end{proof}
\textbf{Corollary:} Let $\ord_n(a)=k$,\\ 
										then $a^1, a^2, \dots, a^k$ are incongruent modulo $n$.
\begin{proof} Assume $1 \leq j \leq i \leq k$\\
										\[ a^i \equiv a^j[n] \]
							The previous theorem implies: \[ i \equiv j[k]\]
																						\[ i=j \]										
\end{proof}
\begin{thm}
						Let $\ord_n(a)=k, h>0$\\
						Then $a^h$ has order $\ord_n(a^h)= \frac{k}{\gcd(h,k)}$\\
\end{thm}
\begin{proof}
							Set $d=\gcd(h,k)$\\
							\[\begin{array}{lcl}
								h=h_1d\\
								k=k_1d\\
								\end{array}
								\text{  and } \gcd(h_1,k_1)=1
								\]
Then: \[ (a^h)^{k_1}=a^{\frac{hk}{d}}=a^{kh_1}\equiv 1^{h_1} \equiv 1[n]\]
Now set $r=\ord_n(a^h)$, $r\mid k_1$\\
			\[a^{hr}=(a^h)^r\equiv 1 [n]\]
			\[k\mid hr \Rightarrow \frac{hr}{k} \in \mathbb{Z} \Rightarrow \frac{\frac{h}{d} r}{\frac{k}{d}} \in \mathbb{Z} \Rightarrow \frac{h_1 r}{k_1} \in \mathbb{Z}\Rightarrow k_1\mid h_1r\]
			\newpage
Euclid's lemma implies: \[ k_1 \mid r \Rightarrow r=k_1\]
												\[\ord_n(a^h)=r=k_1=\frac{k}{d}=\frac{k}{\gcd(h,k)}\]
\end{proof}
\underline{Example:} $n=7$,  $a=3$
										\[\begin{array}{lcl}
										3^2=3 \equiv 2 \nequiv 1[7]\\
										3^3=27 \equiv -1 \nequiv 1[7]\\
										\Rightarrow k=\ord_7(3)6\\
										\end{array}
										\]
					If $h=2$, \[\ord_7(3^2)=\frac{6}{\gcd(2,6)}=\frac{6}{2}=3\]
										\[\text{So }\ord_7(2)=3 \Rightarrow 2^3 \equiv 1[7]\]


Remark: if $a\equiv b [n]$
			\[\begin{array}{lcl}
			b^k\equiv a^k\equiv 1[n]\\
			b^j\equiv a^j\equiv 1[n]
				\end{array} \text{ with } 1 \leq j \leq k-1\]
			$\Rightarrow \ord_n(b)=\ord_n(a)$\\
\begin{defi} $a$ is called a \textbf{primitive root} modulo n if $\ord_n(a)=\varphi(n)$
\end{defi}
\underline{Example:} 3 is a primitive root modulo 7: $\ord_7(3)=6=7-1=\varphi(7)$
\vspace{10mm}\\
\underline{\textbf{Lagrange's theorem:}} Let p be a prime, and $f$ such that:
																					\[f(x)=a_nx^n+\dots+a_1x+a_0\]
																					$a_i\in \mathbb{Z}$\\
																					$a_n \nequiv 0[p]$\\
																					Then  $f(x) \equiv 0[p]$ has at most $n$ incongruent solutions.
\begin{proof} by induction on $n$.
At n=1: \[\begin{array}{lcl}
				&f(x)=&a_1x+a_0\equiv 0[p]\\
				&\Rightarrow& a_1x\equiv -a_0[p]\\
				&&\gcd(a_1,p)=1\\
				&\Rightarrow&\text{ there exist a unique solution modulo p}
				\end{array}\]
				
				Now assume it is true for polygonists of some degree $k-1$ and let $deg(f(x))=k$.
				If $f(x)\equiv0[p]$ has no solution, we're done.
				If such a solution $x=a$ exists,\\ \[f(a)\equiv 0[p]\]
											\[f(x)=(x-a)q(x)+r,\qquad q(x) \in \mathbb{Z}[X], r\in \mathbb{Z}\]
											\newpage
				Since $deg(q(x))=k-1$ \[0 \equiv f(a) \equiv (a-a)q(a)+r=r[p]\]
				Ergo $r\equiv 0[p]$ \[\Rightarrow f(x)=(x-a)q(x)\]
				
				Now suppose $b\neq a$ is another solution to $f(x)\equiv 0[p]$.
															\[0 \equiv f(b) \equiv (b-a)q(b)[p]\]
															\[p \mid (b-a)q(b)\]
															\[b\neq a \Rightarrow p \mid q(b) \Rightarrow q(b) \equiv 0[p]\]
				Since $q(x) \equiv 0[p]$ has less or equal than $k-1$ incongruent solutions, we get that \\
				$f(x) \equiv 0[p]$ has less or equal than $1+(k-1)=k$ incongruent solutions modulo p.
\end{proof}
\textbf{Corollary:} For $p$ prime, $d\mid p-1$: \[x^d-1\equiv0[p]\]
										has exactly $d$ incongruent solutions.
\begin{proof} Let $p-1 = dk$\\
							Then \[ x^{p-1}-1=(x^d-1)\underbrace{(x^{d(k-1)}+\dots+x^{2d}+x^d+1)}_{f(x)}\]
							Fermat's theorem implies: $x^{p-1}-1\equiv 0[p]$ has $p-1$ solutions, and\\
							Lagrange's theorem implies: $f(x)\equiv 0[p]$ has less or equal than $d(k-1)=p-1-d$ solutions.
							Set $a$ a solution of $x^{p-1}-1\equiv 0[p]$:\\
																	\[0\equiv a^{p-1}-1=(a^d-1)f(a)[p]\]
																	\[p \mid (a-1) \quad\text{ or }\quad p \mid f(a)\]
																	\[\Rightarrow \text{ a is a solution of }x^d-1\equiv 0[p]\]
							And by using Lagrange's theorem:\\
							$\left.
							\begin{tabular}{rcr}
										$x^d-1 \equiv 0[p]$ has over or equal to $d$ incongruent solutions[p]\\
										$x^d-1 \equiv 0 [p]$ has less or equal than $d$ incongruent solutions[p]
							\end{tabular}
							\right \}$ 
							$d$ solutions.\\
\end{proof}
\begin{thm} Let $p$ be a prime, $d\mid p-1$, then there are exactly $\varphi(d)$ incongruent integers of order $d$ modulo p.
\end{thm}
\begin{proof} For $d\mid \overbrace{(p-1)}^{=\varphi(p)}$, set $\psi(d)$ the number of $a \leq p-1$ of order $p$.
									\[\Rightarrow p-1=\sum_{d\mid (p-1)} \psi(d)\]
									For $d\mid p-1$ \[ d=\sum_{c\mid d} \psi(c) \; ;\]
									And there are by the corollary exactly d incongruent solutions $a_1, \dots ,a_d$ to $x^d-1 \equiv0[p]$.\\
									Then $a_i^d\equiv 1[p]$.\\
									So $c= \ord_p(a_i)\mid d$
									And if , for some $b \leq p-1$, \[c=\ord_p(b) \mid d\]
																									\[1 \equiv b^c \Rightarrow (b^c)^{dk} \equiv b \equiv 1 [p] \Rightarrow b \in \{ a_1 \dots a_d \}\]
									By M\"{o}bius inversion formula: \[ \Rightarrow \psi(d)=\sum_{c\mid d} \mu(c) \frac{d}{c} = \varphi(c)\]
\end{proof}
\underline{Illustration:} $p=11$\\
\[
\begin{array}{l|cr}
				a & \ord_11(a)  \\
				\hline
				1 & 1 \\
				2 & 10 \\
				3 & 5 \\
				4 & 5 \\
				5 & 5\\
				6 & 10\\
				7 & 10\\
				8 & 10\\
				9 & 5\\
				10 & 2
\end{array}
\left.
\begin{array}{lcr}
			 \psi(1)=1=\varphi(1)\\
			\psi(2)=1=\varphi(2)\\
			\psi(5)=4=\varphi(5)\\
			\psi(1)=1=\varphi(1)\\
\end{array}
\right \} 
10=4+4+1+1									
\]
\textbf{Corollary:} A prime $p$ has exactly $\varphi(p-1)$ promitive elements \\ (This is the case $d=p-1$ of the precedent theorem)\\
\underline{Application:} $p\equiv 1[4] \Rightarrow x^2\equiv -1[p]$ has a solution.\\
													Take $d=4$ in the theorem: $4 \mid p-1$\\
													Then there exists an $a$ of order 4 modulo $p$.\\
													\[p \mid (a^4-1)=(a^2-1)(a^2+1)\]
													\[p \mid a^2-1 \text{  or  } \mid a^2+1\]
													\[\Rightarrow \left. \begin{array}{lcl}
																								a^2 \equiv 1[p]\\
																								a^2 \equiv -1[p]
																								\end{array}
																								\right \} x=a \text{ is a solution to   } x^2 \equiv -1[p]\]
\newpage


{\centering
\Large
\textbf{Number Theory}\\
\normalsize
\textbf{November 19}\\
2014\\
}
\vspace{10mm}
\underline{Lemma:} $p$ an odd prime, there is a primitive element $r [p]$ such that:\\
									\[r^{p-1} \nequiv[p^2]\]
\begin{proof} Let $r$ be a primitive root modulo $p$.\\
							\begin{itemize}
							\item If $r^{p-1} \nequiv 1 [p^2]$, we're done.
							\item If $r^{p-1} \equiv 1 [p^2]$, set $r'=r+p$,$\qquad \qquad \ord_p(r)=\ord_p(r')=p-1$
										\[\begin{array}{lcl}
										&r'^{p-1}&=(r+p)^{p-1}\equiv r^{p-1}+\underbrace{\binom{p-1}{1} r^{p-2}}_{p-1} p[p^2]\\
										&&=r^{p-1}+p^2r^{p-2}-r^{p-2}p\\
										&&\equiv 1-r^{p-2}p\equiv 1[p^2] \text{  since  } p\cancel{\mid} r
											\end{array}\]
							\end{itemize}
\end{proof}
\underline{Lemma:} Let $r$ be a primitive root modulo p such that: \[r^{p-1}\equiv 1[p^2]\]
										Then for each $k\geq 2$, 	\[r^{p^{k-2}(p-1)}\nequiv 1 [p^k]\]
																							\[p^{k-2}(p-1)=\frac{p^{k-1}(p-1)}{p}=\frac{\varphi(p^k)}{p}\]
\begin{proof} Induction on $k\geq 2$\\
							Case $k=2$ is the assumption.
							Now assume for a particular $k$:\[\begin{array}{lcl}
																						r^{p^{k-2}(p-1)}=r^{\varphi(p^{k-1})}\equiv 1[p^{k-1}]\\
																						r^{p^{k-2}(p-1)}=1+ap^{k-1} \quad a\in \mathbb{Z}\quad  p\cancel{\mid}a\\
																						\end{array}\]
							At $k+1$: \[\begin{array}{lcl}
													r^{p^{k-1}(p-1)}=(r^{p^{k-2}(p-1))^p}\equiv(1+ap^{k-1})^p\\
													\equiv 1+\binom{p}{1} ap^{k-1}+\underbrace{\binom{p}{2} a^2p^{2(k-1)}+\dots \qquad}_{=0}\\
													\equiv 1+ap^k[p^{k+1}]\\
													\nequiv 1[p^{k+1}]
													\end{array}\]
\end{proof}																	
\begin{thm} $k\geq 1$, $\forall p$ odd prime, $\exists r[p^k]$ a primitive root.
\end{thm}
\begin{proof} Take $r$ a primitive root modulo p. We assume from a precedent lemma that $r^{p^{k-2}(p-1)} \neq 1[p^k]$
							Set $n=\ord_{p^k}(r)$, then \[\begin{array}{lcl}
																						r^n\equiv 1[p] \Rightarrow (p-1)\mid n\\
																						n\mid\varphi(p^k)=p^{k-1}(p-1)\\
																						\Rightarrow n=p^m(p-1),\qquad 0\leq m \leq k-1\\
																						\end{array}\]
							\begin{itemize}
										\item If $m=k-1$ it's done, $r$ is a primitive root.
										\item If $m \leq k-2$:
													\[r^{p^{k-2}(p-1)}=r^{p^m(p-1)p^{(k-2)-m}} \equiv 1 [p^k] \text{ : absurd!}\]
							\end{itemize}
\end{proof}
\begin{defi} For $\gcd(a,n)=1$, the \textbf{index} of $a$ in the base $r$ is the smallest positive integer $h$ such that:\[a \equiv r^h[n] \]
																																																												\[ind_r(a)=h\]
						If $a\equiv b[n]$ then $ind(a)=ind(b)$
\end{defi}
\begin{thm} $n,r$ as above.
						\begin{itemize}
									\item[a)] $ind_r(a,b)\equiv ind_r(a)+ind_r(b)[\varphi(n)]$
									\item[b)] $ind_r(a^k) \equiv k \cdot ind_r(a)[\varphi(n)]$
									\item[c)] $ind_r(1)\equiv 0$
									\item[d)] $ind_r(r)\equiv 1[\varphi(n)]$
						\end{itemize}			
\end{thm}
\newpage

{\centering
\Large
\textbf{Number Theory}\\
\normalsize
\textbf{November 21}\\
2014\\
}
\vspace{10mm}
\vspace{10mm}
In a polynomial equation, we always state $p\nmid a$, since \\$ax^2+bx+c\equiv 0 [p] \iff bx+c\equiv 0[p]$ if it does.
To introduce the quadratic residue, we will try to prove that \[x^2\equiv a [p] \;\text{, p an odd prime}\] has either 0 or 2 incongruent solutions. Let's suppose $x_0$ is a solution, then $-x_0$ is another:
																																		\[\begin{array}{lcl}
																																							(-x_0)^2\equiv x_0^2 \equiv a[p]\\
																																							\text{If }x_0\equiv -x_0 [p]\\
																																							\Rightarrow 2x_0\equiv 0[p]\\
																																							\Rightarrow p\mid x_0 \\
																																							\Rightarrow p\mid a\text{ :   absurd!}\\
																																							\Rightarrow x_0 \nequiv -x_0 [p]
																																		\end{array}\]
Hence, there are at least 2 solutions. But  we also know by Lagrange's theorem that: $deg(x^2-2)=2 \Rightarrow$ there are 2 or less incongruent solutions.\\
Therefore, there are exactly 2 solutions.\\

																																		
\begin{defi} $p$ an off prime, $p\mid a$. $a$ is called a quadric residue modulo $p$ if \[x^2 \equiv a[p] \text{  has 2 incongruent solution}\]
						$a$ is called a quadric non residue if the equation has no solution.
\end{defi}
\underline{Example:} $p=11$
										\[\left.
										\begin{array}{lcl}
											1^2 \equiv 10^2 \equiv 1\\
											2^2 \equiv 9^2 \equiv 4 \\
											3^2 \equiv 8^2 \equiv 9 \\
											4^2 \equiv 7^2 \equiv 5 \\
											5^2 \equiv 6^2 \equiv 3
											\end{array}
											\right\} [11]\]
										$1,3,4,5,9$ are quadric residues, and $2,6,7,8,10$ are quadric non residues [11].
\vspace{5mm}										
\underline{\textbf{Euler's criterion:}} Let $p$ be an odd prime, and $p \nmid a$. \\Then $a$ is a quadric residue iff:\[a^{\frac{p-1}{2}}\equiv 1[p]\]
\begin{proof}:\\
							\underline{$\Rightarrow$} Let $x_1$ be such that $x_1^2 \equiv a [p]$\\
							\[a^\frac{p-1}{2} \equiv (x_1^2)^\frac{p-1}{2}=x_1^{p-1}\underbrace{\equiv 1[p]}_{\scriptscriptstyle \text{Fermat's thm}}\]
							\underline{$\Leftarrow$} Assume $a^\frac{p-1}{2}\equiv 1[p]$\\
							Fix a primitive root $r[p]$ and 
																							\[\begin{array}{lcl}
																							a\equiv r^k[p], \; k=ind_p(a)\\
																							1\equiv a^\frac{p-1}{2}\equiv (r^k)^\frac{p-1}{2}=r^\frac{k(p-1)}{2}[p]\\
																							\ord_p(r)=(p-1)\mid\frac{k(p-1)}{2}\\
																							\end{array}\]
							Therefore $\exists \ell \in \mathbb{Z} /\;\frac{k(p-1)}{2}=(p-1)\ell\implies k=2\ell$ is even and $a=r^k=r^{2\ell}[p]$ is a quadric residue!		
\end{proof}
\newpage
\underline{"Extra argument"}
														\[\begin{array}{lcl}
														a^{p-1}\equiv 1[p]\\
														p\mid(a^{p-1}-1)=(a^\frac{p-1}{2}-1)(a^\frac{p-1}{2}+1)\\
														\Rightarrow a^\frac{p-1}{2}\equiv \pm 1[p]\\
														\end{array}\]
														Hence $a$ is a quadric non residue iff $a^\frac{p-1}{2}\equiv -1[p]$
\begin{defi} $p$ an odd prime, $p\nmid a$.\\
						 The \textbf{Lagrange symbol} is defined by:\[(a/p)=\left \{
																																\begin{array}{lcl}
																																&1&\text{ if } a \text{ is a quadric residue mod }p\\
																																&-1&\text{ if } a \text{ is a quadric non residue mod }p\\
																																\end{array}
																																\right.\]
\end{defi}
\underline{Example:} \[\begin{array}{lcl}(1/11)=(4/11)=(9/11)=(5/11)=(3/11)=1\\
																				 (2/11)=(6/11)=(7/11)=(8/11)=(10/11)=-1
											\end{array}\]
														
\begin{thm} $p$ an odd prime, $p\nmid a$, $p\nmid b$. Then:
						\begin{itemize}
						\item[a)] $a\equiv b[p] \Rightarrow (a/p)=(b/p)$
						\item[b)] $(a^2/p)=1$
						\item[c)] $(a/p)\equiv a^\frac{p-1}{2}[p]$
						\item[d)] $(ab/p)\equiv (a/p)(b/p)$
						\item[e)] $(1/p)=1 \qquad (-1/p)=(-1)^\frac{p-1}{2}$
						\end{itemize}
\end{thm}
The proofs are quite easy so they will not figure here.

\textbf{Corollary:}\[(-1/p)=\left \{ \begin{array}{lcl}
																			1\\
																			-1\\
																			\end{array}
																			\right. \iff 
																			\begin{array}{lcl}
																			p\equiv 1[4]\\
																			p\equiv 3[4]\\
																			\end{array}\]
\begin{proof}
							\[p\equiv1[4] \Rightarrow p=4k+1, p\equiv 3[4] \implies p=4k+3\]
							\[\text{Since }(-1/p)=(-1)^\frac{p-1}{2}\text{, we have  }
																											\left \{
																											\begin{array}{lcl}
																											(-1)^{2k}=1\\
																											(-1)^{2k+1}=-1
																											\end{array}
																											\right.
							\]
\end{proof}
\underline{Example:}   $(76/43)\overset{(a)}{=}(-10/43)\overset{(d)}{=}(-1/43)(2/43)(5/43)=(-1)\cdot 1 \cdot 1=-1$
\begin{thm}There are infinitely many primes of the form $4k+1$\end{thm}
\begin{proof}: Assume $p_1 \dots p_n$ are all the primes $\equiv 1[4]$, and set:
							\[ N=(2p_1p_2\dots p_n)^2+1\]
							Let $p$ be a factor in $N$, then $p$ is odd since $N$ is odd.
												\[\begin{array}{lcl}
													p\mid N, \; N\equiv 0[p]\\
													(2p_1p_2\dots p_n)^2+1\equiv 0[p]\\
													\Rightarrow (2p_1\dots p_n)^2\equiv -1[p]\\
													\Rightarrow (-1/p)=1\\
													\Rightarrow p\equiv 1[4]\\
													\Rightarrow p \mid N-(2p_1\dots p_n)=1\text{: absurd!}
													\end{array}
													\]
													Hence there exists infinitely many primes $\equiv 1[4]$.
\end{proof}						
						

\newpage


{\centering
\Large
\textbf{Number Theory}\\
\normalsize
\textbf{November 25}\\
2014\\
}
\vspace{10mm}
\underline{\textbf{Gauss's lemma:}} $p$ an odd prime, $p\nmid a$\\
																		Let $n$ denote the number of elements in \[ S=\{a,2a,\dots,({\scriptstyle \frac{p-1}{2}})a\}\] whose remainders [p] lie in $(\frac{p}{2},p)$, then: \[(a/p)=(-1)n\]
\begin{proof} Denote the remainders:   $0<r_1<\dots<r_m<\frac{p}{2}<s_1<\dots<s_n<p$\\
							If we set $m=\mid\{r_i\}\mid$, and $n=\mid \{s_j\} \mid$,   $m+n=\frac{p-1}{2}$\\
							$(r_1,\dots r_m, p-s_1,\dots,p-s_n)$are distincts and exhaust $\{1,2,\dots,\frac{p-1}{2}\}$.\\
							Now assume $p-s_i=r_j$ for some $i,j$.
							\[\exists  n,r \in \mathbb{Z}/\;1\leq u,v \leq \frac{p-1}{2} \]
							\[\begin{array}{lcl}\left.									
																	\begin{array}{rrr}
																	r_j\equiv va[p]\\
																	s_i\equiv ua[p]\\
																	\end{array}
																	\right \}
																	&\Rightarrow& (u+v)a\equiv v_j+s_i=p\equiv 0[p]\\
																	&\Rightarrow& p\mid (u+v) \; \text{ or }\; p\mid a \text{ (contradiction with the initial conditions)}\\
																	&\Rightarrow& p\mid (u+v) \text{ but } 0<2\leq (u+v) \leq p-1 < p\\
								\end{array}\]
								So $p\nmid (u+v)$.
								\[\begin{array}{rrl}
								&(\frac{p-1}{2})!&=r_1\dots r_m(p-s_1)\dots(p-s_n)\\
																&&=r_1\dots r_m(-s_1)\dots(-s_n)\\
																&&=(-1)^nr_1\dots r_ms_1\dots s_n\\
																&&=(-1)^n(a(2a)\cdots(\frac{p-1}{2})a\\
																&&\equiv(-1)^n(\frac{p-1}{2})!a^\frac{p-1}{2}[p]\qquad \text{ and } \gcd(\frac{p-1}{2},p)=1\text{  so:}\\
															&1& \equiv (-1)^na^\frac{p-1}{2}[p]\\
															\Rightarrow &(-1)^n&\equiv a^\frac{p-1}{2}[p]\\
									\end{array}\]
								And by Euler's criterion:\[(a/p)\equiv a^\frac{p-1}{2}\equiv (-1)^n[p]\]
																					\[\Rightarrow (a/p) \in \{\pm 1\}\]
								Since $p>2$, the congruence must be an equality.
\end{proof}
\underline{Example:} $p=13$, $a=5$\\
										 \[S=\{5,10,15,20,25,30\}\equiv \{5,10,2,7,12,4\}\]
										 \[(5/13)=(-1)^5=-1 \text{ so 5 is a non residue }[13]\]

\begin{defi}$\lfloor x \rfloor$is the largest integer less or equal to $x$.\end{defi}
\begin{thm} $p$ an odd prime, then:
						\[(2/p)=\left \{
										\begin{array}{lcl}
										1 \text{ if } p\equiv \pm 1[8]\\
										-1 \text{ if } p\equiv \pm 3[8]\\
										\end{array}
										\right.\]
\end{thm}
\begin{proof}$(2/p)=(-1)^n$, with n the number of remainders in $(\frac{p}{2},p)$, from $S=\{2,4,6,\dots,(\frac{p-1}{2})2\}$.
							\[\begin{array}{l|c|c}
							p&\frac{p-1}{2}-\lfloor \frac{p}{4} \rfloor = n &(-1)^n=(2/p)\\
							\hline
							8k+1 & 4k-2k=2k & 1 \\
							8k+3 & 4k+1-2k=2k+1 & -1\\
							8k+5 & 4k+2-(2k+1)=2k+1&-1\\
							8k+7 & 4k+3-(2k+1)=2k+2&1\\
							\end{array}\]
\end{proof}
\textbf{Corollary:}$(2/p)=(-1)^\frac{p^2-1}{8}$
\begin{proof} \hspace{3mm}\\ \begin{itemize}
							\item If $p=8k\pm 1$\\
										$\Rightarrow \frac{p^2-1}{8}=8k^2\pm 2$ is even.
							\item If $p=8k\pm 3$\\
										$\Rightarrow \frac{p^2-1}{8}=8k^2\pm 6+1$ is odd.
							\end{itemize}
\end{proof}
\begin{thm}$p$ an odd prime.
						\[\sum_{a=1}^{p-1} (a/p)=0\]
						i.e. there are exactly $\frac{p-1}{2}$ quadric residues and $\frac{p-1}{2}$ quadric non residues $[p]$.\\
\end{thm}
\begin{proof} Let $r$ be a primitive root modulo $p$, then:
							\[\begin{array}{lcl}
											\{1,2,\dots,p-1\}\equiv\{r,r^2,\dots,r^{p-1}\}[p]\\
											\sum_{a=1}^{p-1} (a/p)=\sum_{k=1}^{p-1} (r^k/p)=\sum_{k=1}^{p-1} (r/p)^k=\sum_{k=1}^{p-1}(-1)^k\\
											(r/p)\overbrace{\equiv}^{\scriptscriptstyle \text{Euler's crit}} r^\frac{p-1}{2}\equiv (-1)[p]\\
											(r/p)=-1\\
								\end{array}\]
							And since $p-1$ is even, the sum vanishes.
\end{proof}

\newpage


{\centering
\Large
\textbf{Number Theory}\\
\normalsize
\textbf{November 28}\\
2014\\
}
\vspace{10mm}
\underline{\textbf{Lemma:}} Let $p$ be an odd prime, $a$ an odd integer, $p \nmid a$.\\
Then, $(a/p)=(-1)^{\sum_{k=1}^{\frac{p-1}{2}} \lfloor \frac{ka}{p} \rfloor}$	
\begin{proof} Consider $S=\{a,2a,\dots,(\frac{p-1}{2})a\}$.
											 \[ka = q_kp+t_k \qquad \text{with}\quad 1\leq t_k\leq p-1 \:,\quad 1 \leq k \leq \frac{p-1}{2}\]
											 \[\{t_1 \dots t_{\frac{p-1}{2}}\}=\{r_i\}_{i=1}^m+\{s_j\}_{j=1}^n\]
											 \[\frac{ka}{p}=q_k+\frac{t_k}{p} \qquad \qquad 0<\frac{t_k}{p}<1 \Rightarrow q_k= \lfloor \frac{ka}{p} \rfloor\]
							The we calculate the two sums:\begin{itemize}
											\item \[\sum_{k=1}^{\frac{p-1}{2}} ka = \sum_{k=1}^{\frac{p-1}{2}} \lfloor \frac{ka}{p} \rfloor + \sum_{i=1}^{m}r_i + \sum_{j=1}^{n}s_i\]
											\item And considering $\{r_i, p-s_j\}=\{t_1,t_2,\cdots,\frac{p-1}{2}\}$ (from a previous result),
																						\[\sum_{k=1}^{\frac{p-1}{2}} k = \sum_{i=1}^{m} r_i+ \sum_{j=1}^{n}(p-s_j)=np+\sum{ri}+\sum{sj}\]
											\end{itemize}
							We then substract: \[ \sum_{k=1}^{\frac{p-1}{2}} ak - \sum_{k=1}^{\frac{p-1}{2}}k = p(\sum_{k=1}^{\frac{p-1}{2}}\lfloor \frac{ka}{p} \rfloor - n)+ 2 \sum sj\]
							And looking at that modulo 2, with $a$ and $p$ both still odd: \[ 0 \equiv \sum_{k=1}^{\frac{p-1}{2}} \lfloor \frac{ka}{p} \rfloor -n [2]\]
																																							\[\Rightarrow \sum_{k=1}^{\frac{p-1}{2}} \lfloor \frac{ka}{p} \rfloor \equiv n [2] \qquad \text{ but then} \quad (a/p)=(-1)^n=(-1)^{\sum \lfloor \frac{ka}{p} \rfloor}\]
\end{proof}
\underline{\textbf{Gauss's quadratic reciprocity theorem:}}\\
									$p\neq q$ two odd primes, then: \[(p/q)(q/p)=(-1)^{\frac{p-1}{2}\frac{q-1}{2}}\]
									The exact proof is in the book.
\textbf{Corollary:} $p\neq q$ two odd primes, then \[(p/q)(q/p)=\left \{ 
																																\begin{array}{lcl}
																																1 \text{ if }\: p \:\text{or} \:q \equiv 1[4]\\
																																-1 \text{ if }\: p \:\text{or} \:q \equiv 3[4]\\
																																\end{array} 
																																\right.\]
\textbf{Corollary:} $p\neq q$ two odd primes, then	\[(p/q)=\left\{ 
																														\begin{array}{lcl}
																																(q/p) \text{ if }\: p \:\text{or} \:q \equiv 1[4]\\
																																-(q/p) \text{ if }\: p \:\text{or} \:q \equiv 3[4]\\
																														\end{array}
																														\right.\]
\underline{Example:} \[(37/89)\Rightarrow \left. 
																				\begin{array}{lcl}
																				37 \equiv 1[4]\\ 
																				89 \equiv 15 [37]
																				\end{array}
																				\right \} \Rightarrow (15/37)=(3/37)(5/37)=(1/3)(2/5)=1\cdot -1 = -1\]
%\underline{Example:} Let $p>3$ be a prime, then:\[(3/p) = \left \{ \begin{array}{lcl}
%																																		(p/3) \text{if} \: p \equiv 1[4]\\
%																																		-(p/3) \text{if} \: p \equiv 3[4]\\
%																																		\end{array} \right.\]
%$p=1$ or $2[3]$ \[\Rightarrow p\equiv 1[3] \Rightarrow (p/3)	=(1/3)=1\]
%								\[p\equiv 2[3] = (p/3)=(2/3)=-1
%%%%%%%%%%%%%%% I GOT LAZY.
\begin{thm} If $p\neq 3$ is an odd prime, then \[ (3/p)=\left\{
																												\begin{array}{lcl}
																												1 if p\equiv \pm 1 [12]\\
																												-1 if p\equiv \pm 5 [12]\\
																												\end{array}
																												\right.\]
\end{thm}

																																				


\newpage



{\centering
\Large
\textbf{Number Theory}\\
\normalsize
\textbf{December 2}\\
2014\\
}
\vspace{10mm}
\underline{Question:} What integer can be written as a sum of 2 squares?\\
\underline{\textbf{Lemma:}} \[\text{If  }\left\{
																					\begin{array}{lcl}
																					m=a^2+b^2\\
																					n=c^2+d^2
																					\end{array}
																					\right. \Rightarrow mn \;\text{is also a sum of two squares.}\]
\begin{proof} \[\text{Set  }\left\{
														\begin{array}{lcl}
														z=a+bi\\
														w=c+di
														\end{array}
														\right.
														\left\{
														\begin{array}{lcl}
														m=a^2+b^2=|z|^2\\
														n=c^2+d^2=|w|^2
														\end{array}
														\right.
							\]							
							\[\begin{array}{lcl}
								(a^2+b^2)(c^2+d^2)=mn=|z|^2|w|^2=|zw|^2\\
								=|(a+bi)(c+di)|^2\\
								=|(ac-bd)+(ad+bc)i|\\
								=(ac-bd)^2+(ad+bc)^2
								\end{array}
							\]
\end{proof}
\underline{\textbf{Dirichlet's box principle:}} If $n$ objects are placed in $m$ boxes, and $n>m$, then some boxes contains more than one object.\\


\underline{Thue's Lemma:} $p$ a prime, $a\in \mathbb{Z}$, $p\nmid a$\\
Then $ax\equiv y [p]$ has a solution $x_0,y_0 \in \mathbb{Z}$/ 
																		\[0<|x_0|<\sqrt{p},\qquad 0<|y_0|<\sqrt{p}\]
\begin{proof} Set $k=\lfloor\sqrt{p} \rfloor +1$, and consider:
																		\[\begin{array}{rcl}
																			f:\{0,1,\dots,k-1\} \times \{0,1,\dots,k-1\} &\longrightarrow& \{0,1,\dots,q-1\}\\
																			(x,y)&\longrightarrow& ax-y[p]
																			\end{array}\]
							As $k>p^2$, Dirichlet's box principle implies: $\exists (x_1,y_1)\neq(x_2,y_2)$/
																														\[\begin{array}{lcl}
																														f(x_1,y_1)=f(x_2,y_2)\\
																														ax_1-y_1=ax_2-y_2[p]\\
																														ax_1-ax_2=y_1-y_2[p]\\
																																		\left (
																																		\begin{array}{lcl}
																																		x_0=x_1-x_2\\
																																		y_0=y_1-y_2
																																		\end{array}
																																		\right )
																																		\rightarrow ax_0=y_0[p]
																														\end{array}\]
We prove easily that $x_0$ and $y_0$ are both non zero.
\end{proof}


\newpage
\underline{\textbf{Fermat's theorem:}} An odd prime $p$ is a sum of 2 squares iff $p\equiv 1[4]$.
\begin{proof}:\\
								\underline{$\Rightarrow$} Assume \[p=a^2+b^2\qquad a,b \in \mathbb{N}\]
								Then $p\nmid a$, for if it does: \[ a=pk\implies p=a^2+b^2\geq a^2=p^2k \geq p^2\]
								Symetrically, $p\nmid b.$\\
								Then, $\exists c \in \mathbb{Z}/ \quad bc\equiv 1[p]$ ,   and:
								
																										\[\begin{array}{lcl}
																										c^2\mid (ac)^2+(bc)^2=(a^2+b^2)c^2=pc^2\equiv 0[p]\\
																										\Rightarrow (ac)^2+1\equiv 0[p]\\
																										ac^2\equiv -1[p]\\
																										(-1/p)=1 \Rightarrow p\equiv 1[4]
																										\end{array}\]
\underline{$\Leftarrow$} Let $p\equiv 1[4]$ \[\Rightarrow (-1/p)=1\]
																						\[\Rightarrow a^2\equiv -1[p]\]
												Then $a\nequiv 0[p]$ so $p\nmid a$ and Thue's lemma says $\exists x,y \in \mathbb{Z}$/
																								\[ax\equiv y[p]\]
																								\[\begin{array}{ccc}
																								0<|x|<\sqrt{p},\quad 0<|y|<\sqrt{p}\\
																								y^2\equiv (ax)^2=a^2x^2\equiv -x^2[p]\\
																								\Rightarrow x^2+y^2\equiv 0[p]\\
																								x^2+y^2=kp \quad \text{for some p} \in \mathbb{Z}\\
																								0<x^2+y^2<p+p=2p\\
																								\Rightarrow k=1, \;x^2+y^2=p\\
																								\end{array}\]
\end{proof}
\textbf{Proposition:} $p$ a prime of the form $4k+1$ can be represented uniquely as a sum of two sqares.
\begin{proof} Assume $p= a^2+b^2=c^2+d^2$ where $a,b,c,d$ are positive integers.\\
Then \[ \begin{array}{lcl}
				&a^2d^2-b^2d^2&=a^2d^2+b^2d^2-b^2d^2-b^2c^2\\
				&&=d^2(a^2+b^2)-b^2(c^2+d^2)\\
				&&=d^2p-b^2p\equiv 0[p]
				\end{array}\]
				So $p\mid(ad+bc)$ or $p \mid (ad-bc)$.
				Now \[\begin{array}{lcl}
							0<a,b,c,d<\sqrt{p} &\Rightarrow& 0<ad,bc<p\\
							\Rightarrow ad=bc \quad &\text{ or}& \quad ad=p-bc
							\end{array}\]
\newpage
				In this last case, $p=ad+bc$, so:\[p^2=(a^2+b^2)(a^2+d^2)=(ad+bc)^2+(ac-bd)^2=p^2+(ac-bd)^2\]
																					\[\begin{array}{lcl}
																						&\Rightarrow& ac-bd=0\\
																						&&ac=bd
																						\end{array}\]
				Then we have either$ ad=bc$ or $ac=bd$.\\
				By symetry $(c\rightarrow d)$ we assume $ad=bc$.\\
				If $\gcd(a,b)>1$, then \[\gcd(a,b)^2\mid  a^2+b^2=p^2\] Which is absurd, therefore, $a$ and $b$ are relatively prime.
															\[\begin{array}{lcl}
															a\mid ad=bc \Rightarrow a\mid c\\
															c=ka,\quad \text{for some }k \in \mathbb{Z}\\
															\end{array}\]
				\[\begin{array}{lcr}
				ad=bka\\
				d=bk\\
				(c,d)=k(a,b)\\
				p=c^2+d^2=(ka)^2+(kb)^2=k^2(a^2+b^2)=k^2p\\
				\Rightarrow k^2=1\\
				\Rightarrow k=1\\
				\Rightarrow (c,d)=(a,b)
				\end{array}\]
				
\end{proof}


\underline{Example:} $p=29 \equiv 1[4]$; $a^2\equiv-1[29]$
															\[a=12;\qquad 12^2=144\equiv -1[29]\]
															\[12x\equiv y[29]\qquad 145=5\cdot 29\]
															\[0<|x|,|y|<\sqrt{29}\]
															\[\begin{array}{r|cl}
																			x&12x\equiv y\\
																			\hline
																		1&12\equiv 12\\
																		2&24\equiv -5\\
																		3&36\equiv 7\\
																		4&48\equiv -10\\
																		5&50\equiv 2
															\end{array}\]
															\[\begin{array}{lcl}
																(x,y)=(2,-5)\\
																12\cdot 2\equiv -5[29]\\
																5^2=y^2\equiv (12x)^2\equiv -1x^2\equiv -2^2[29]\\
																5^2+2^2\equiv 0[29]\\
																5^2+2^2=29
																\end{array}
															\]
\begin{thm} Let $n\in \mathbb{N}$, $n=N^2m$, with m square free.\\
						Then $n=a^2+b^2 \iff m$ contains no prime factor of the form $4k+3$
\end{thm}
\begin{proof}:\\
			\underline{$\Rightarrow$} Assume \[n=a^2+b^2=N^2m\]
													Let $p$ be an off prime, $p\mid m$.
													Set: \[\begin{array}{lcl}
																				d=\gcd(a,b)\\
																				a=dr\\
																				b=ds
																	\end{array}\text{  with }\gcd(r,s)=1
																	\]
																	\[d^2(r^2+s^2)=(dr)^2+(ds)^2=a^2+b^2=n=N^2m\]
																	\[\Rightarrow d^2\mid N^2 \;\text{as }m\text{ is square free?}\]
																	\[r^2+s^2\equiv 0[p]\]
																	\[\gcd(r,s)=1\equiv p\nmid r \; \text{or} \; p\nmid s\]
																	By symetry, $p\nmid r$.\\
																	$\exists r' \in \mathbb{Z}$/ \[\begin{array}{lcl}
																																				rr'\equiv 1[p]\\
																																				(rr')^2+(sr')\equiv 0[p]\\
																																				1+(sr')^2\equiv 0[p]\\
																																				(sr')^2\equiv -1[p]\\
																																				\Rightarrow (-1/p)=1\\
																																				\Rightarrow p\equiv 1[4]
																																	\end{array}\]
			\underline{$\Leftarrow$} If $m=1$, $n=N^2$.\\
																Let $m>1$, $m=p_1\dots p_r$ (distincts primes).
																$p_i\equiv 1$ or $2 [4]$, $\exists x_i, y_i \in \mathbb{Z}$/
																\[p_i=x_i^2+y_i^2\qquad 1\leq i \leq r\]
																By repeatedly using multiplicativity:
																\[\begin{array}{lll}
																	m=p_1\dots p_r=x^2+y^2\\
																	n=N^2m=N^2(x^2+y^2)=(Nx)^2+(Ny)^2=a^2+b^2
																	\end{array}\]

\end{proof}


\underline{Example:}	\begin{itemize}
											\item \[459=3^3\cdot 17=\underbrace{3^2}_{N^2}\underbrace{(3\cdot 17)}_m\]
														$3=3[4]$ so $n=459\neq a^2+b^2$.
											\item \[n=153=3^2\cdot 17\]
														$m=4^2+1^2$\\
														\[153=3^2(4^2+1^2)=(3\cdot 4)^2+3^2=12^2+3^2\]
											\end{itemize}
																
\begin{thm} <<Officially part of the course but not so important>> $n\in \mathbb{Z}$ can be written $n=a^2-b^2 \iff n\nequiv 2[4]$
\end{thm}


\newpage



{\centering
\Large
\textbf{Number Theory}\\
\normalsize
\textbf{December 5}\\
2014\\
}
\vspace{10mm}
\subsection*{Fibonacci sequence}
$	\left \{
	\begin{array}{lcl}
		u_1=u_2=1\\
		u_n=u_{n-1}+u_{n-2}
	\end{array}
	\right.$\\
	
	
\underline{Claim:} $u_{5n+2}>10^n$ for $n\geq1$.
\begin{proof} Induction on $n$: \[ n=1\qquad u_7=13>10^1\]
							Now assume it is true for some $n$:
							\[\begin{array}{lcl}
										&u_{5n+2}>10^n&\\
										&u_{5(n+1)+2}&=u_{5n+7}=u_{5n+6}+u_{5n+5}\\
													&&			=2(u_{5n+5}+u_{5n+3})+u_{5n+4}\\
													&&			=3(u_{5n+3}+u_{5n+2})+2u_{5n+3}\\
													&&			=5(u_{5n+2}+u_{5n+1})+3u_{5n+2}\\
													&&			\begin{array}{lcl}
																		=&8_{5n+2}+5u_{5n+1}&>8u_{5n+2}+2\underbrace{(u_{5n+1}+u_{5n})}_{=u_{5n+2}}\\
																		&&								 =10u_{5n+2}>10\cdot 10^n=10^{n+1}\\	
																		\end{array}
							\end{array}\]
\end{proof}
\begin{thm} $\gcd(u_n,u_{n+1})=1 \qquad \forall n \geq 1$
\end{thm}
\begin{proof}
							Set $d=\gcd(u_n,u_{n+1})$\\
							\[\begin{array}{lcl}
								\Rightarrow d \mid u_{n+1}-u_n=u_{n-1}\\
								\Rightarrow d\mid u_n-u_{n-1}=u_{n-1}\\
								\Rightarrow \quad \vdots\\
								\Rightarrow d\mid u_3-u_2=1\\
								\Rightarrow d=1
								\end{array}\]
\end{proof}


\underline{Proposition:} for $m\geq 2$, $n \geq 1$: \[u_{m+n}=u_{m-1}u_n+u_mu_{n+1}\]
\begin{proof} Fix $m\geq 2$. Induction on $n$:
														\[\begin{array}{lcl}
														\fbox{n=1}\qquad &u_{m+1}&=u_{m-1}u_1+u_mu_2\\
																										&&=u_{n-1}+u_m
															\end{array}\]
							Now assume it is true for all integer until some $n$. We show it is true for $n+1$:
														\[\begin{array}{lcl}
														\text{at } n-1\quad &u_{m+n-1}&=u_{m-1}u_{n-1}+u_mu_n\\
														\text{at } n  \quad &u_{m+n}  &=u_{m-1}u_n+u_mu_{n+1}\\
														\text{at } n+1\quad &u_{m+n+1}&=u_{m-1}u_{n-1}+u_mu_n+u_{m-1}u_n+u_mu_{n+1}\\
																												&& \begin{array}{lcl}
																														=u_{m-1}(u_n+u_{n-1})+u_m(u_n+u_{n+1})\\
																														=u_mu_{n+1}+u_mu_{k+2}\\
																														\end{array}
														\end{array}\]
\end{proof}																									
\begin{thm} For $m\geq 1$, $n \geq 1$ \[u_m\mid u_{mn}\]
\end{thm}
\begin{proof} Fix $m\geq 1$ and by induction on $n$:\\
									\[\fbox{n=1} \qquad u_m\mid u_{m\cdot 1}=u_m\]
							Now assume $u_m\mid u_{mn}$ for some $n$, and consider the case $n+1$:
									\[u_{m(n+1)}=u_{mn+n}=u_{mn-1}u_m+u_{mn}u_{m+1}\]
									\[\begin{array}{lcl} \Rightarrow u_m\mid u_m \; ;\qquad u_m \mid u_{mn} &\Rightarrow& u_m \mid u_{mn-1}u_m+u_{mn}u_{m+1}\\
																																												&\Rightarrow& u_m \mid u_{m(n+1)}
										\end{array}\]
\end{proof}
\underline{\textbf{Lemma:}} Let $m=qn+r$, with $m,n,q,r \geq 1$, then:
														\[\gcd(u_m,u_n)=\gcd(u_r,u_n)\]
\begin{proof}
			\[\begin{array}{lcl}
				&\gcd(u_m,u_n)&=\gcd(u_{qn+r},u_n)\\
				&&=\gcd(u_{qn-1}u_r+u_{qn}u_{r+1},u_n)
				\end{array}\]
			\[\left [
				\text{We know that:} \quad u_n\mid u_{qn} \implies \gcd(a+bk,b)=\gcd(a,b)
				\right ] \]
			\[\qquad \;=\gcd(u_{qn-1}u_r,u_n)\]
			We now try to prove: \[\gcd(u_{qn-1},u_n)=1\]
			If $d=\gcd(u_{qn-1},u_n)$ then $d\mid u_n \mid u_{qn} \Rightarrow 	\left\{
																																				\begin{array}{lcl}
																																				d \mid u_{qn-1}\\
																																				d \mid u_{qn}
																																				\end{array}
																																				\right. \Rightarrow d=1$
\end{proof}
\begin{thm} The $\gcd$ of two Fibonacci numbers is a Fibonacci number: \[\gcd(u_m,u_n)=u_{\gcd(m,n)}\]
\end{thm}
\begin{proof} Assume $m\geq n$. We use the euclidian algorithm:
							\[\begin{array}{lcl}
								m=q_1n+r_1 \qquad && \qquad 0<r_1<n\\
								n=q_2r_1+r_2\qquad && \qquad 0<r_2<r_1\\
									\qquad	\vdots  &&   \qquad \vdots\\
								r_{k-2}=q_kr_{k-1}+r_k&&\\
								r_{k-1}=q_{k-1}r_k+0\qquad  && \qquad 0 <r_k<r_{k-1}
								\end{array}
								\vline
								\begin{array}{lcl}
												\gcd(u_m,u_n)=\gcd(u_{r_1},u_n)\\
												=\gcd(u_n,u_{r_1})=\gcd(u_{r_2},u_{r_1})\\
												=\dots =\gcd(u_{r_{k-1}},u_{r_k})\\
												\text{And since  } r_k\mid r_{k-1} \Rightarrow u_{r_k}\mid u_{r_{k-1}}\\
												\text{We have } u_{\gcd(m,n)}
								\end{array}\]
\end{proof}

\newpage


{\centering
\Large
\textbf{Number Theory}\\
\normalsize
\textbf{December 9}\\
2014\\
}
\vspace{10mm}
\begin{defi} A finite continued fraction is an expression: \[a_0+\frac{1}{a_1+\frac{1}{a_2+\frac{1}{\begin{array}{lcl}
																																																		&\scriptstyle{\vdots}&\scriptstyle{\qquad \vdots \qquad \vdots}\\
																																																		&&\scriptstyle{a_{n-1}+\frac{1}{a_n}}
																																																		\end{array}}}}\]
\end{defi}
\begin{thm} Any rational number can be written as a finite simple continued fraction.\end{thm}
\begin{proof} in the book p.308. \end{proof}

\underline{Example:} \[\begin{array}{lcl}
												\frac{a}{b} = \frac{43}{13}\\
												43=3\cdot 13 +4\\
												13=3\cdot 4+1\\
												4=4\cdot 1
												\end{array}
												\vline
												\begin{array}{lcl}
												\frac{43}{13} = 3+\frac{4}{13}=3+\frac{1}{\frac{13}{4}}\\
												\frac{13}{4}=3+\frac{1}{4}=3+\frac{1}{4}{1}\\
												\frac{4}{1}=4
												\end{array}
												\]
												\[\Rightarrow \frac{43}{13}=3+\frac{1}{3+\frac{1}{4}}\]
												\[\Rightarrow \frac{13}{43}=0+\frac{1}{3+\frac{1}{3+\frac{1}{4}}}\]
												
												
												
Notation: $[a_0;a_1,a_2,\dots,a_n]=a_0+\frac{1}{a_1+\frac{1}{\begin{array}{lcl}
																															&\scriptstyle{\vdots}& \qquad \scriptstyle{\vdots \qquad \vdots}\\
																															&&\scriptstyle{a_{n-1}+\frac{1}{a_n}}
																															\end{array}}}$
					\[\frac{13}{43}=[0;3,3,4]\]
\begin{defi} $C_k=[a_0;a_1,a_2,\dots,a_k]_{1\leq k\leq n}$;  is called the $k$\up{th} convergent to $[a_0;a_1,\dots,a_k,\dots,a_n]$.
\end{defi}
\underline{Example:} \[	\left.
												\begin{array}{lcl}
												[0;3,3,4]\\
												C_0=0\\
												C_1=0+\frac{1}{3}=\frac{1}{3}\\
												C_2=0+\frac{1}{3+\frac{1}{3}}=\frac{3}{10}\\
												C_3=0+\frac{1}{3+\frac{1}{3+\frac{1}{4}}}=\frac{13}{43}
												\end{array}
												\right \} C_k=\frac{p_k}{q_k}\text{, with  }
												\left [
												\begin{array}{lll}
												p_0=a_0&q_0=1\\
												p_1=a_1a_0+1&q_1=a_1\\
												p_k=a_kp_{k-1}&q_k=a_kq_{k-1}+q_{k-2}\\
												\end{array}
												\right ]
												\]
\begin{thm} The $k$\up{th} convergent $C_k=\frac{p_k}{q_k}$ with $p_k, q_k$ given by recursion.
\end{thm}
\begin{proof} We prove this more generally for $a_0,a_1,\dots,a_n \in \mathbb{R}.$\\
							Induction on $k$: True for $C_0=\frac{p_0}{q_0},\quad C_1=\frac{p_1}{q_1},\quad C_2=\frac{p_2}{q_2}$.\\
							Now assume it's true for some $m$ such that: $2\leq m \leq n$: \[[a_0;a_1,\dots,a_m]\;=\;\frac{p_m}{q_m}=\frac{a_mp_{m-1}+p_{p-2}}{a_mq_{m-1}+q_{m-2}}\]
							Since $p_{m-1},p_{m-2},q_{m-2},q_{m-2}$ only depend on $a_0,a_1,\dots,a_{m-1}$ but not $a_m$:
																																							\[\begin{array}{lcl}
																																									&C_{m+1}&=[a_0;a_1,\dots,a_m,a_{m+1}]=[a_0;a_1,\dots,a_m+\frac{1}{a_{m+1}}]\\
																																									&&=\frac{(a_m+\frac{1}{a_{m+1}})p_{m-1}+p_{m-2}}{(a_m+\frac{1}{a_{m+1}})q_{m-1}+q_{m-2}} \cdot \frac{a_{m-1}}{a_{m-1}}\\
																																									&&=\frac{a_{m+1}(a_mp_{m-1}+p_{m-2})+p_{m-1}}{a_{m+1}(a_mq_{m-1}+q_{m-2})+q_{m-1}}\\
																																									&&=\frac{p_{m+1}}{q_{m+1}}
																																								\end{array}\]
							Therefore, the theorem holds for $m+1$ and for all $m\leq n$ by finite induction.
\end{proof}


\underline{Convention:}
											\[\left \{
												\begin{array}{lcl}
												p_{-2}=0&q_{-2}=1\\
												p_{-1}=1&q_{-1}=0
												\end{array}
												\right.
												\left(
												\begin{array}{l|l|l|l|l|l|l}
												k&-2&-1&0&1&2&3\\ \hline
												a_k&.&.&0&3&3&4\\
												p_k&0&1&0&1&3&13\\
												q_k&1&0&1&3&10&43\\
												C_k&.&.&0&\frac{1}{3}&\frac{3}{10}&\frac{13}{43}
												\end{array}
												\right)
											\]
\begin{thm}The convergents $\frac{p_k}{q_k}\qquad \scriptstyle{\forall k\leq n}$ satisfy:\[p_kq_{k-1}-p_{k-1}q_k=(-1)^{k-1}\]
\end{thm}
\begin{proof} \[
							\begin{bmatrix}
								p_k & q_k\\
								p_{k-1}& q_{k-1}
							\end{bmatrix}
							=
							\begin{bmatrix}
								a_k&1\\
								1&0
							\end{bmatrix}
							\begin{bmatrix}
								p_{k-1}&q_{k-1}\\
								p_{k-2}&q_{k-2}
							\end{bmatrix}
							\]
							If me set $M_k=\begin{bmatrix}
															p_k&q_k\\
															p_{k-1}&q_{k-1}
															\end{bmatrix}$
															; and $A_k=
															\begin{bmatrix}
																a_k&1\\
																1&0
															\end{bmatrix}$, then: $M_k=A_kM_{k-1}$.	
And therefore: \[A_kM_{k-1}=A_k(A_{k-1}M_{k-2})=\dots\dots=A_kA_{k-1}\dots A_0M_{-1}\]
								\[M_{-1}=\begin{bmatrix}
													p_{-1}&q_{-1}\\
													p_{-1}&q_{-2}
													\end{bmatrix}
													=
													\begin{bmatrix}
													1&0\\
													0&1
													\end{bmatrix}
									=Id\]
													
								\[\begin{array}{lcl}
								&\Rightarrow det(M_k)=p_kq_{k-1}-q_kp_{k-1}=\prod_{j=0}^{k} det(A_j)det(M_{-1})&	=(-1)^{k+1}\\
																																															&&	=(-1)^{k-1}
																																																\end{array}\]
\end{proof}

\newpage



{\centering
\Large
\textbf{Number Theory}\\
\normalsize
\textbf{December 9}\\
2014\\
}
\vspace{10mm}
\underline{\textbf{Lemma:}}  $q_{k-1}\leq q_k$ for $A\leq k \leq n$ with a strict inequality for $k>1$.\\
														Hence $q_k\longrightarrow \infty$ as $k\longrightarrow \infty$.
\begin{proof}: $k=1$, $q_k=1\leq a_1=q_1,$  and let $k>1$, then:\[q_k=a_kq_{k-1}+q_{k-2}>a_kq_{k-1}\geq q_{k-1}\]
\end{proof}
\begin{thm} The convergents satisfy:
						\begin{itemize}
						\item[a)] $C_0<C_2<C_2<\dots$
						\item[b)] $C_1<C_3<C_5<\dots$
						\item[c)] $C_{2s}<C_{2r-1} \quad \forall s\geq 0, r\geq 1.$
						\end{itemize}
\end{thm}
\begin{proof}
							\[\begin{array}{lcl}
								&C_{k+2}-C_k&=(C_{k+2}-C_{k+1})+(C_{k+1}-C_k)\\
													&& =(\frac{p_{k+2}}{q_{k+2}}-\frac{p_{k+1}}{q_{k+1}})+(\frac{p_{k+1}}{q_{k+1}}-\frac{p_n}{q_n})\\
													&& =\frac{(-1)^{k+1}}{q_{k+2}q_{k+1}}+\frac{(-1)^k}{q_{k+1}q_k}=(-1)^k\frac{q_{k+2}-q_k}{q_{k+2}q_{k+1}q_k}\\
													&& \Rightarrow 	\left\{
																					\begin{array}{lcl}
																					\text{If k is even :} C_{k+2}-C_k>0 \Rightarrow C_k<C_{k+2} \; &(a)&\\
																					\text{If k is odd :} C_{k+2}-C_k<0\Rightarrow C_{k+2}<C_k \; &(b)&
																					\end{array}
																					\right.
								\end{array}\]
\end{proof}
\begin{defi}An infinite continued fraction is an expression:
						\[[a_0;a_1,a_2,\dots,a_n,\dots]=a_0+\frac{1}{a_1+\frac{1}{a_2+\frac{1}{a_3+\frac{1}{\vdots}}}}\]
With $aj \in \mathbb{Z}$  $aj>0$ for $j\geq 1$\\
This has a convergent $C_n$: \[C_n=[a_0;a_1,a_2,\dots,a_n]=\lim\limits_{n\to \infty}C_n\]
\underline{Proof of the convergence:} \[C_0<C_2<C_4<\dots<C_5<C_3<C_1\]
																			Set \[\left.
																					\begin{array}{lcl}
																					\alpha = \lim\limits_{n \to infty} C_{2n}\\
																					\alpha'= \lim\limits_{n \to infty} C_{2n-1}
																					\end{array}
																					\right \}
																					\alpha \leq \alpha'\]
																					\[0 \leq \alpha - \alpha' < C_{2n+1}-C_{2n}= \frac{p_{2n+1}}{q_{2n+1}} - \frac{p_{2n}}{q_{2n}}=\frac{(-1)^{2n}}{q_{2n+1}q_{2n}}=\underbrace{\frac{1}{q_{2n+1}q_{2n}}}_{\rightarrow 0 \text{ when } n \to \infty} \]
\end{defi}



\underline{Example:} Consider $[1;1,1,1,\dots,1,\dots] \quad a_n=1 \forall n.$
										 \[p_n=p_{n-1}+p_{n-2}\quad q_n=q_{n-1}+q_{n-2} \]
											\[\begin{array}{l|rrrrrrrr}
												k&-2 &-1 &0 &1 &2 &3 &4 &5\\
												\hline
												p_k&0 &1 &1 &2 &3& 5 &8 &13\\
												q_k&1 &0 &11& 2& 3& 5& 8\\
												\end{array}\]
												\[\Rightarrow C_n=\frac{u_{n+2}}{u_{n+1}} \text{ since it's the Fibonnacci sequence.}\]
												\[\Rightarrow \lim\limits_{n\to \infty}C_n=\lim\limits_{n\to \infty}\frac{u_{n+2}}{u_{n+1}}=\lim\limits_{n\to \infty} \frac{u_{n+2}}{u_{n+1}}=\lim\limits_{n\to \infty} \frac{u_{n+1}+u_n}{u_{n+1}}=\lim\limits_{n\to \infty} 1+\frac{1}{\frac{u_{n+1}}{u_n}}=\lim\limits_{n\to \infty}1+\frac{1}{\lim\limits_{n\to \infty} \frac{u_{n+1}}{u_n}}\]
												\[\Rightarrow \lim\limits_{n\to \infty} \frac{u_{n+2}}{u_{n+1}}=\frac{\lim\limits_{n\to \infty}\frac{u_{n+2}}{u_{n+1}} +1}{\lim\limits_{n\to \infty}\frac{u_{n+2}}{u_{n+1}}}\Rightarrow \lim\limits_{n\to \infty} C_n = \frac{\lim\limits_{n\to \infty} C_n+1}{\lim\limits_{n\to \infty} C_n}
												\Rightarrow (\lim\limits_{n\to \infty} C_n)^2 = \lim\limits_{n\to \infty} C_n +1\]
												\[\Rightarrow x^2-x+1=0 \iff x = \frac{1\pm \sqrt{5}}{2} \text{ but } x_0>C_0 \Rightarrow \frac{1+\sqrt{5}}{2}\]
												\[\Rightarrow\frac{1+\sqrt{5}}{2}=[1;1,1,\dots]\]
												
												
\underline{Notation:} We write: \[[3;1,2,1,6,1,2,1,6,1,\dots]=[3;\bar{1,2,1,6}]\]
																
\begin{thm} The value of an infinite continued fraction is irrational.\end{thm}
\begin{proof} Set $x=[a_0,a_1,\dots,a_n,\dots]=\lim\limits_{n\to \infty} C_n=\lim\limits_{n\to \infty} \frac{p_n}{q_n}$\\
							Let $n\geq 0$. As $x$ lies between $C_n$ and $C_{n+1}$ \[0<|x-C_n|<|C_{n+1}-C_n|=|\frac{p_{n+1}}{q_{n+1}}-\frac{p_n}{q_n}|=\frac{1}{q_{n+1}q_n}\]
							Now, assume (for a contradiction) that $x=\frac{a}{b} \in \mathbb{Q}$:
																								\[0<|\frac{a}{b}-\frac{p_n}{q_n}|<\frac{1}{q_{n+1}q_n}\]
\end{proof}
												





\end{document}																								